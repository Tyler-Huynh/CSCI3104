\documentclass[11pt]{article} 
\usepackage[english]{babel}
\usepackage[utf8]{inputenc}
\usepackage[margin=0.5in,top=0.5in,bottom=0.75in]{geometry}
\usepackage{amsmath}
\usepackage{amsthm}
\usepackage{amsfonts}
\usepackage{amssymb}
\usepackage[usenames,dvipsnames]{xcolor}
\usepackage{graphicx}
\usepackage[siunitx]{circuitikz}
\usepackage{tikz}
\usepackage[colorinlistoftodos, color=orange!50]{todonotes}
\usepackage{hyperref}
\usepackage[numbers, square]{natbib}
\usepackage{fancybox}
\usepackage{epsfig}
\usepackage{soul}
\usepackage[framemethod=tikz]{mdframed}
\usetikzlibrary{positioning, automata, backgrounds}
\usepackage{tikz}\usetikzlibrary{arrows.meta,backgrounds,calc,quotes}
\usepackage[shortlabels]{enumitem}
\usepackage[version=4]{mhchem}
\usepackage{multicol}
\usepackage{forest}
\usepackage{mathtools}
\usepackage{comment}
\usepackage{enumitem}
\usepackage[utf8]{inputenc}
% \usepackage[linesnumbered,ruled,vlined]{algorithm2e}
\usepackage{algorithm}
\usepackage[noend]{algpseudocode}
\usepackage{listings}
\usepackage{color}
\usepackage[numbers]{natbib}
% \usepackage{algorithm}
\usepackage{subfiles}
\usepackage{tkz-berge}


\newtheorem{prop}{Proposition}[section]
\newtheorem{thm}{Theorem}[section]
\newtheorem{lemma}{Lemma}[section]
\newtheorem{cor}{Corollary}[prop]
\DeclarePairedDelimiter\ceil{\lceil}{\rceil}
\DeclarePairedDelimiter\floor{\lfloor}{\rfloor}

\theoremstyle{definition}
\newtheorem{definition}{Definition}

\theoremstyle{definition}
\newtheorem{required}{Problem}
\newtheorem*{requiredHC}{Problem HC}

\theoremstyle{definition}
\newtheorem{ex}{Example}

\newcommand{\interval}[4]{\draw (#2, #1) -- (#3, #1); % Usage: \interval{height}{start}{end}{label}
\draw (#2, #1-0.11) -- (#2, #1+0.11); % draw left whisker
\draw (#3, #1-0.11) -- (#3, #1+0.11); % draw right whisker
\node[] at (#2-0.25, #1) {#4};
}

\tikzset{>={Stealth[length=7pt]}}
\tikzset{
    vertex/.style={circle,draw,minimum size=16,inner sep=0pt,font=\normalsize},
    edgelabel/.style={rectangle,draw=none,font=\footnotesize,outer sep=0pt},
    every node/.style={vertex},
    every edge quotes/.append style={edgelabel},
    every to/.append style={every node/.style={edgelabel}},
    wide/.style={line width=4pt,>={Stealth[length=18pt]}},
    directed/.style={arrows={->},font=\small},
    caption/.style={text width=6cm,align=center,rectangle,draw},
}


\setlength{\marginparwidth}{3.4cm}
%#########################################################

%To use symbols for footnotes
\renewcommand*{\thefootnote}{\fnsymbol{footnote}}
%To change footnotes back to numbers uncomment the following line
%\renewcommand*{\thefootnote}{\arabic{footnote}}

% Enable this command to adjust line spacing for inline math equations.
% \everymath{\displaystyle}

% _______ _____ _______ _      ______ 
%|__   __|_   _|__   __| |    |  ____|
%   | |    | |    | |  | |    | |__   
%   | |    | |    | |  | |    |  __|  
%   | |   _| |_   | |  | |____| |____ 
%   |_|  |_____|  |_|  |______|______|
%%%%%%%%%%%%%%%%%%%%%%%%%%%%%%%%%%%%%%%

\title{
\normalfont \normalsize 
\textsc{CSCI 3104 Fall 2022 \\ 
Instructors: Prof. Grochow and Nagesh} \\
[10pt] 
\rule{\linewidth}{0.5pt} \\[6pt] 
\huge Problem Set 6 \\
\rule{\linewidth}{2pt}  \\[10pt]
}
%\author{}
\date{}

\begin{document}

\definecolor{processblue}{cmyk}{0.96,0,0,0}
\definecolor{processred}{rgb}{200, 0, 0}
\definecolor{processgreen}{rgb}{0, 255, 0}
\DeclareGraphicsExtensions{.png}
\DeclareGraphicsExtensions{.gif}
\DeclareGraphicsExtensions{.jpg}

\maketitle


%%%%%%%%%%%%%%%%%%%%%%%%%
%%%%%%%%%%%%%%%%%%%%%%%%%%
%%%%%%%%%%FILL IN YOUR NAME%%%%%%%
%%%%%%%%%%AND STUDENT ID%%%%%%%%
%%%%%%%%%%%%%%%%%%%%%%%%%%
\noindent
Due Date \dotfill October 17, 2022 \\
Name \dotfill \textbf{Tyler Huynh} \\
Student ID \dotfill \textbf{109603994} \\
Collaborators \dotfill \textbf{N/A}

\renewcommand{\thesubsection}{\arabic{section}(\alph{subsection}). }

\tableofcontents

\section*{Instructions}
\addcontentsline{toc}{section}{Instructions}
 \begin{itemize}
	\item The solutions \textbf{must be typed}, using proper mathematical notation. We cannot accept hand-written solutions. Useful links and references on \LaTeX can be found \href{https://canvas.colorado.edu/courses/75824/pages/latex}{here on Canvas}.
	\item You should submit your work through the \textbf{class Canvas page} only. Please submit one PDF file, compiled using this \LaTeX \ template.
	\item You may not need a full page for your solutions; pagebreaks are there to help Gradescope automatically find where each problem is. Even if you do not attempt every problem, please submit this document with no fewer pages than the blank template (or Gradescope has issues with it).

	\item You are welcome and encouraged to collaborate with your classmates, as well as consult outside resources. You must \textbf{cite your sources in this document.} \textbf{Copying from any source is an Honor Code violation. Furthermore, all submissions must be in your own words and reflect your understanding of the material.} If there is any confusion about this policy, it is your responsibility to clarify before the due date. 

	\item Posting to \textbf{any} service including, but not limited to Chegg, Reddit, StackExchange, etc., for help on an assignment is a violation of the Honor Code.

	\item You \textbf{must} virtually sign the Honor Code (see Section \hyperlink{HonorCode}{Honor Code}). Failure to do so will result in your assignment not being graded.
\end{itemize}


\section*{Honor Code (Make Sure to Virtually Sign the Honor Pledge)} 
\addcontentsline{toc}{section}{Honor Code (Make Sure to Virtually Sign the Honor Pledge)}
\hypertarget{HonorCode}{}

\begin{requiredHC}
On my honor, my submission reflects the following:
\begin{itemize}
\item My submission is in my own words and reflects my understanding of the material.
\item Any collaborations and external sources have been clearly cited in this document.
\item I have not posted to external services including, but not limited to Chegg, Reddit, StackExchange, etc.
\item I have neither copied nor provided others solutions they can copy.
\end{itemize}

\noindent In the specified region below, clearly indicate that you have upheld the Honor Code. Then type your name. 
\end{requiredHC}

\begin{proof}[Honor Pledge]
I, \textbf{Tyler Huynh} on my honor pledge that my submission is a reflection of my own understanding of the material, any and all collaborations/sources have been properly cited, I have not posted any material to external sources, and I have not copied other solutions as my own.
\end{proof}



\newpage
\setcounter{section}{15}
\section{Standard 16 - Analyzing Code: Writing Down Recurrences}

\setcounter{required}{15}
\begin{required} \label{wdr}

For each algorithm, write down the recurrence relation for the number of times they print ``Welcome''. (In this case, this is big-$\Theta$ of the runtime - do you see why?) \textbf{Don't forget to include the base cases.}

\begin{enumerate}[label=(\alph*)]
\subsection{Problem 16\ref{wdr1}}
\item \label{wdr1}

\begin{algorithm}
\caption{Writing Recurrences 1}\label{alg:Recurrence1}
\begin{algorithmic}[1]
\Procedure{Foo}{$\text{Integer } n$}
\If{$n \leq 5$} 
\Return n
\EndIf

\noindent \\
\State $\text{Foo}(n/10)$
\State $\text{Foo}(n/5)$
\State $\text{Foo}(n/5)$ \\
    
\For{$i \gets 1; i \leq 3 * n; i \gets i+1 $}
    \State \textbf{print} \text{``Welcome"}
\EndFor
\EndProcedure
\end{algorithmic}
\end{algorithm}

\begin{proof}[Answer] \textbf{Referenced Levet Notes} \\
\textbf{Base Case:} \\
For the base case it will just return n, such that it will be: \\
\begin{center}
$T(0)$ = $\Theta(1)$ \\
$T(1)$ = $\Theta(1)$ \\ 
$T(2)$ = $\Theta(1)$ \\
$T(3)$ = $\Theta(1)$ \\
$T(4)$ = $\Theta(1)$ \\ 
$T(5)$ = $\Theta(1)$ \\
\end{center}
From the above that we can see that for the base case when it returns n it will take $\Theta(1)$  to run. \\

\textbf{Recursive Case:} \\
\indent-On line 8 for the recursive case the function will take 1 step to initialize i at 1 \\
\indent-It will also take 2 steps, 1 step  to evaluate $i + 1$ and 1 more step for the assignment of $i \gets$ on line 8 \\
\indent-It will take 1 step to print "Welcome" on line 9 \\

The function will run a total 3n times. \\
From this we can find the total runtime complexity of the for loop: \\
\begin{align*}
1 + \sum_{i=1}^{3n} (1 + 2 + 1) &= 1 + \sum_{i=1}^{3n}  4 \\
&= 1+ 12n \\
&= \Theta(n)
\end{align*}

I will now find the runtime complexity of the function in general: \\
\begin{align*}
T(n) &= T(\frac{n}{10}) + T(\frac{n}{5}) + T(\frac{n}{5}) + \Theta(n)\\
&= 2T(\frac{n}{5}) + T(\frac{n}{10}) + \Theta(n)
\end{align*}

$$
T(n) = \begin{cases}
\Theta(1) & : n \leq 5, \\
2T(\frac{n}{2}) + T(\frac{n}{10}) + \Theta(n) & : n > 5.
\end{cases}
$$

From the above this is final answer for the runtime complexity of the function.
\end{proof}

\newpage
\subsection{Problem 16\ref{wdr2}}
\item \label{wdr2}

\begin{algorithm}
\caption{Writing Recurrences 2}\label{alg:Recurrence2}
\begin{algorithmic}[1]
\Procedure{Foo2}{$\text{Integer } n$}
\If{$n \leq 7$} 
\Return
\EndIf

\noindent \\
\State $\text{Foo2}(n/14)$
\State $\text{Foo2}(n/14)$ \\
    
\For{$i \gets 1; i \leq n; i \gets i * 2 $}
    \For{$j \gets 1; j \leq n; j \gets j * 3 $}
        \State \textbf{print} \text{``Welcome"}
    \EndFor
\EndFor
\EndProcedure
\end{algorithmic}
\end{algorithm}

\begin{proof}[Answer]
\textbf{Base Case:} \\
For the base case it will just, such that: \\
\begin{center}
$T(0)$ = $\Theta(1)$ \\
$T(1)$ = $\Theta(1)$ \\ 
$T(2)$ = $\Theta(1)$ \\
$T(3)$ = $\Theta(1)$ \\
$T(4)$ = $\Theta(1)$ \\ 
$T(5)$ = $\Theta(1)$ \\
$T(6)$ = $\Theta(1)$ \\
$T(7)$ = $\Theta(1)$ \\
\end{center}
From the above we can see that for the base case when it returns n it will take $\Theta(1)$ to run. \\

\textbf{Recursive Case:} \\
Inner Loop: \\
\indent -We see that initializing j at 1 will take 1 step on line 8\\
\indent -We see that the evaluation of j * 3 will take 1 step and the 1 more step for the assignment of $j \gets$ \\
\indent -We see that the print statement "Welcome" will take 1 step \\
\indent -The inner loop will run $\log_3 n$ times

Outer Loop: \\
\indent -We see that initializing i at 1 will take 1 step on line 7\\
\indent -We see that the evaluation of i * 2 will take 1 step and the 1 more step for the assignment of $i \gets$ \\
\indent -For the rest of the steps it will exist within the inner loop of j \\
\indent -The outer loop will run $\log_2 n$ times



I will now find the runtime complexity of the inner loop: \\
\begin{align*}
1 + \sum_{j=1}^{\log_3 n} (1 + 2 + 1) &= 1 + \sum_{j=1}^{\log_3 n}  4 \\
&= 1 + 4(\log_3 n)\\
&= 1 + 4\log_3 n\\
&= \Theta(\log n) \\
\end{align*}

Runtime complexity of the outer loop: \\
\begin{align*}
1 + \sum_{i=1}^{\log_2 n} (1 + 2) + (1 + (4\log_3 n))&= 1 + \sum_{i=1}^{\log_2 n}  (1 + 2) +  (1 + (4\log_3 n))\\
&=1 + \sum_{i=1}^{\log_2 n} 4 + (4\log_3 n)\\
&= 1 + \sum_{i=1}^{\log_2 n} 4 + 4\sum_{i=1}^{\log_2 n} \log_3 n \\
&= 1 + 4(\log_2 n) + 4(\log_2 n)(\log_3 n) \qquad \log_3 n = \frac{\log_2 n}{\log_2 3}\\
&= 1 + 4\log_2 n + 4(\log_2 n)(\frac{\log_2 n}{\log_2 3}) \\
&= 1 + 4\log_2 n + (\frac{4}{\log_2 3})((\log_2 n)^2) \\
&= \Theta((\log n)^2)
\end{align*}
From the above we have found the run time of the two loops, I will now write out our recurrence relation: \\
\begin{align*}
T(n) &= T(\frac{n}{14}) + T(\frac{n}{14}) + \Theta((\log_n)^2)\\
&= 2T(\frac{n}{14}) + \Theta((\log_n)^2)
\end{align*}
Thus, we have our final runtime complexity of the function: \\
$$
T(n) = \begin{cases}
\Theta(1) & : n \leq7, \\
2T(\frac{n}{14}) + \Theta((\log_n)^2) & : n > 7.
\end{cases}
$$





\end{proof}

\end{enumerate}

\end{required}

\newpage
\section{Standard 17 - Solving Recurrences I: Unrolling}

\begin{required} \label{unrolling}
For each of the following recurrences, solve them using the unrolling method (i.e. find a suitable function $f(n)$ such that $T(n) \in \Theta(f(n))$). \textbf{Note: } Show all the work.

\renewcommand{\theenumi}{\alph{enumi}}
\begin{enumerate}
\subsection{Problem 17(a)}
\item 
$$
T(n) = \begin{cases}
3n & : n < 3, \\
2T(n / 2) + 4n & : n \geq 3.
\end{cases}
$$

\begin{proof}[Answer]
I will first begin by finding how many times we must unroll up to k: \\
\begin{center}
We will hit a base case when: \\
$\frac{n}{2^k} < 3$
I will now solve for k such that: \\
$k > \log_2 (\frac{n}{3})$ \\
\end{center} 

I will now begin the unrolling process: \\
\begin{align*}
T(n) &= 2T(\frac{n}{2}) + 4n  \\
T(\frac{n}{2}) &= 2T(\frac{\frac{n}{2}}{2}) + 4(\frac{n}{2})  \\
&= 2T(\frac{n}{4}) + 4(\frac{n}{2}) \\
T(\frac{n}{4}) &= 2T(\frac{\frac{n}{4}}{2}) + 4(\frac{n}{4})  \\
&= 2T(\frac{n}{8}) + 4(\frac{n}{4}) \\
\end{align*}

From the above I will now 

\end{proof}


\newpage
\subsection{Problem 17(b)}
\item 
\begin{align*}
T(n) &= \begin{cases} 
5 & : n < 2, \\
7T(n-2) + 9 & : n \geq 2.
\end{cases}
\end{align*}

\begin{proof}[Answer] 
 We will now find how many times we must unroll first up to k: \\
\begin{center}
We will hit a base case when: \\
$n-2k < 2$
I will now solve for k such that: \\
$k > \frac{n-2}{2}$
\end{center}
From the piecewise function we see that 2 will be subtracted multiple times in a row up to k, such that: \\
\begin{align*}
T(n-2) &= n - 2 - 2 - 2 - 2 - 2 - 2 - 2 \\
n - 2 - 2 - 2 - 2 - 2 - 2 - 2 &= k \\
\end{align*}
From this we can find the steps of $T(n)$, such that: \\
\begin{align*}
T(n) &= 7T(n-2) + 9  \\
T(n-2) &= 7T(n-4) + 9 \\
T(n-4) &= 7T(n-6) + 9 \\
T(n-6) &= 7T(n-8) + 9 \\
\end{align*}
From this we can then find the function such that $T(n) \in \Theta(f(n))$: \\
\begin{align*}
T(n) &= 7T(n-2) + 9  \\
&= 7[7T(n-4) + 9] +9 \\
&= 7[7[7T(n-6) + 9] +9] + 9 \\
&= 7^3(T(n-6)) +7^2(9) + 7^1(9) + 7^0(9)
\end{align*}
To save us time we can sum this data up into a summation such that: \\
\begin{align*}
T(n) &= \sum_{i=0}^{\frac{n-2}{2}} 7^{i} * 9 \\
\end{align*}

\begin{align*}
T(n) &= \textrm{(cost of base case) · (number of times the base case is reached)} + \sum_{i=0}^{\frac{n-2}{2}} 7^{i} * 9 \\
&= 5 * 7^{\frac{n-2}{2}} + 9 *  \sum_{i=0}^{\frac{n-2}{2}} 7^{i} \\
&= 5 * 7^{\frac{n-2}{2}} + 9 * \frac{1 - 7^{\frac{n-2}{2}+1}}{1-7} \\
&= 5 * 7^{\frac{n-2}{2}} + \frac{9 - 9*7^{\frac{n-2}{2}+\frac{2}{2}}}{-6} \\
&= 5 * 7^{\frac{n-2}{2}}  + \frac{9-9*7^{\frac{n}{2}}}{-6} \\
\end{align*}
From the above we can observe that the highest component of the function is: \\
\begin{align*}
T(n) &= \Theta(7^{\frac{n}{2}})
\end{align*}
Thus, my final answer is $T(n) = \Theta(7^{\frac{n}{2}})$
\end{proof}
\end{enumerate}
\end{required}

\newpage
\section{Standard 18 - Divide and Conquer: Counterexamples}

\begin{required} \label{dandc}
Consider the following problem:
\begin{quotation}
\noindent \textsc{Max Pair Sum}

\noindent \textit{Input:} A list $L$ of integers

\noindent \textit{Output:} An index $i \in \{1,\dotsc,len(L)-1\}$ such that $L[i]+L[i+1]$ is maximized (that is, such that $L[i] + L[i+1] \geq L[j] + L[j+1]$ for all $j$), and the value of $L[i]+L[i+1]$
\end{quotation}
(Note the list here is 1-indexed, so the problem simply does not consider the last element, as it has nothing to pair it with.)

Consider the algorithm below that attempts to solve this problem. \textbf{Give an instance} of input (preferably a list of length at most 6) for which it fails to output the correct value for the above problem, and \textbf{explain why it fails}.

\begin{algorithm}
\caption{Proposed divide-and-conquer algorithm for the Max Pair Sum problem}
\begin{algorithmic}[1]
\Procedure{MaxPairSum}{$\text{List } L$}
$n \gets len(L)$
\If{$n \leq 1$}
\Return;
\EndIf

\If{$n = 2$} 
\Return $(1,L[1] + L[2])$;
\EndIf


\State $(i1,sum1) \gets \text{MaxPairSum}(L[1..\lfloor n/2 \rfloor])$;

\State $(i2,sum2) \gets \text{MaxPairSum}(L[\lfloor n/2 \rfloor + 1..n])$;

\If{$i1 \geq i2$}

\State \Return $(i1,sum1)$;
\Else

\State \Return $(i2,sum2)$;

\EndIf
\EndProcedure
\end{algorithmic}
\end{algorithm}

\begin{proof}
% YOUR ANSWER HERE
\end{proof}
\end{required}



%%%%%%%%%%%%%%%%%%%%%%%%%%%%%%%%%%%%%%%%%%%%%%%%%%

\end{document} % NOTHING AFTER THIS LINE IS PART OF THE DOCUMENT
