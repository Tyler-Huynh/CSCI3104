\documentclass[11pt]{article} 
\usepackage[english]{babel}
\usepackage[utf8]{inputenc}
\usepackage[margin=0.5in]{geometry}
\usepackage{amsmath}
\usepackage{amsthm}
\usepackage{amsfonts}
\usepackage{amssymb}
\usepackage[usenames,dvipsnames]{xcolor}
\usepackage{graphicx}
\usepackage[colorinlistoftodos, color=orange!50]{todonotes}
\usepackage{hyperref}
\usepackage[numbers, square]{natbib}
\usepackage{fancybox}
\usepackage{epsfig}
\usepackage{soul}
\usepackage[framemethod=tikz]{mdframed}
\usepackage[shortlabels]{enumitem}
\usepackage[version=4]{mhchem}
\usepackage{multicol}
\usepackage{forest}
\usepackage{mathtools}
\usepackage{comment}
\usepackage{enumitem}
\usepackage[utf8]{inputenc}
\usepackage{listings}
\usepackage{color}
\usepackage[numbers]{natbib}
\usepackage{subfiles}
\usepackage{algorithm}
\usepackage[noend]{algpseudocode}


\newtheorem{prop}{Proposition}[section]
\newtheorem{thm}{Theorem}[section]
\newtheorem{lemma}{Lemma}[section]
\newtheorem{cor}{Corollary}[prop]

\theoremstyle{definition}
\newtheorem{definition}{Definition}

\theoremstyle{definition}
\newtheorem{required}{Problem}
\newtheorem*{requiredHC}{Problem HC}

\theoremstyle{definition}
\newtheorem{ex}{Example}

\newcommand{\interval}[4]{\draw (#2, #1) -- (#3, #1); % Usage: \interval{height}{start}{end}{label}
\draw (#2, #1-0.11) -- (#2, #1+0.11); % draw left whisker
\draw (#3, #1-0.11) -- (#3, #1+0.11); % draw right whisker
\node[] at (#2-0.25, #1) {#4};
}


\setlength{\marginparwidth}{3.4cm}
%#########################################################

%To use symbols for footnotes
\renewcommand*{\thefootnote}{\fnsymbol{footnote}}
%To change footnotes back to numbers uncomment the following line
%\renewcommand*{\thefootnote}{\arabic{footnote}}

% Enable this command to adjust line spacing for inline math equations.
% \everymath{\displaystyle}

% _______ _____ _______ _      ______ 
%|__   __|_   _|__   __| |    |  ____|
%   | |    | |    | |  | |    | |__   
%   | |    | |    | |  | |    |  __|  
%   | |   _| |_   | |  | |____| |____ 
%   |_|  |_____|  |_|  |______|______|
%%%%%%%%%%%%%%%%%%%%%%%%%%%%%%%%%%%%%%%

\title{
\normalfont \normalsize 
\textsc{CSCI 3104 Fall 2022 \\ 
Prof. Grochow and Nagesh} \\
[10pt] 
\rule{\linewidth}{0.5pt} \\[6pt] 
\huge Problem Set 5 \\
\rule{\linewidth}{2pt}  \\[10pt]
}
%\author{Your Name}
\date{}

\begin{document}
\definecolor {processblue}{cmyk}{0.96,0,0,0}
\maketitle


%%%%%%%%%%%%%%%%%%%%%%%%%
%%%%%%%%%%%%%%%%%%%%%%%%%%
%%%%%%%%%%FILL IN YOUR NAME%%%%%%%
%%%%%%%%%%AND STUDENT ID%%%%%%%%
%%%%%%%%%%%%%%%%%%%%%%%%%%
\noindent
Due Date \dotfill Monday October 10, 2022 8pm MT \\
Name \dotfill \textbf{Tyler Huynh} \\
Student ID \dotfill \textbf{109603994} \\
Collaborators \dotfill \textbf{List Your Collaborators Here}

\tableofcontents

\section*{Instructions}
\addcontentsline{toc}{section}{Instructions}
 \begin{itemize}
	\item The solutions \textbf{must be typed}, using proper mathematical notation. We cannot accept hand-written solutions. Useful links and references on \LaTeX can be found \href{https://canvas.colorado.edu/courses/75824/pages/latex}{here on Canvas}.
	\item You should submit your work through the \textbf{class Canvas page} only. Please submit one PDF file, compiled using this \LaTeX \ template.
	\item You may not need a full page for your solutions; pagebreaks are there to help Gradescope automatically find where each problem is. Even if you do not attempt every problem, please submit this document with no fewer pages than the blank template (or Gradescope has issues with it).

	\item You are welcome and encouraged to collaborate with your classmates, as well as consult outside resources. You must \textbf{cite your sources in this document.} \textbf{Copying from any source is an Honor Code violation. Furthermore, all submissions must be in your own words and reflect your understanding of the material.} If there is any confusion about this policy, it is your responsibility to clarify before the due date. 

	\item Posting to \textbf{any} service including, but not limited to Chegg, Reddit, StackExchange, etc., for help on an assignment is a violation of the Honor Code.

	\item You \textbf{must} virtually sign the Honor Code (see Section \hyperlink{HonorCode}{Honor Code}). Failure to do so will result in your assignment not being graded.
\end{itemize}


\section*{Honor Code (Make Sure to Virtually Sign the Honor Pledge)} 
\addcontentsline{toc}{section}{Honor Code (Make Sure to Virtually Sign the Honor Pledge)}
\hypertarget{HonorCode}{}

\begin{requiredHC}
On my honor, my submission reflects the following:
\begin{itemize}
\item My submission is in my own words and reflects my understanding of the material.
\item Any collaborations and external sources have been clearly cited in this document.
\item I have not posted to external services including, but not limited to Chegg, Reddit, StackExchange, etc.
\item I have neither copied nor provided others solutions they can copy.
\end{itemize}

\noindent In the specified region below, clearly indicate that you have upheld the Honor Code. Then type your name. 
\end{requiredHC}

\begin{proof}[Honor Pledge]
I, \textbf{Tyler Huynh} on my honor pledge that my submission is a reflection of my own understanding of the material, any and all collaborations/sources have been properly cited, I have not posted any material to external sources, and I have not copied other solutions as my own.
\end{proof}


\newpage

\setcounter{section}{11}
\setcounter{required}{11}
\section{Standard 12- Asymptotics I (Calculus I techniques)}
\begin{required}
For each part, you will be given a list of functions. Your goal is to order the functions from \textbf{slowest growing} to \textbf{fastest growing}. That is, if your answer is $f_{1}(n), \ldots, f_{k}(n)$, then it should be the case that $f_{i}(n) \leq O(f_{i+1}(n))$ for all $i$. If two adjacent functions have the same order of growth (that is, $f_{i}(n) = \Theta(f_{i+1}(n))$), clearly specify this. \textbf{Show all work, including Calculus details.} Plugging into WolframAlpha is not sufficient. \\

\noindent You may find the following helpful.
\begin{itemize}
\item Recall that our asymptotic relations are transitive. So if $f(n) \leq O(g(n))$ and $g(n) \leq O(h(n))$, then $f(n) \leq O(h(n))$. The same applies for little-o, Big-Theta, etc. Note that the goal is to order the growth rates, so transitivity is very helpful. We encourage you to make use of transitivity rather than comparing all possible pairs of functions, as using transitivity will make your life easier.

\item You may also use the Limit Comparison Test. However, you \textbf{MUST} show all limit computations at the same level of detail as in Calculus I-II. Should you choose to use Calculus tools, whether you use them correctly will count towards your score.

\item You may \textbf{NOT} use heuristic arguments, such as comparing degrees of polynomials or identifying the ``high order term" in the function.

\item If it is the case that $g(n) = c \cdot f(n)$ for some constant $c$, you may conclude that $f(n) = \Theta(g(n))$ without using Calculus tools. You must clearly identify the constant $c$ (with any supporting work necessary to identify the constant- such as exponent or logarithm rules) and include a sentence to justify your reasoning. 
\end{itemize}


\noindent You may also find it helpful to order the functions using an \texttt{itemize} block, with the work following the end of the \texttt{itemize} block.
\begin{itemize}
\item This function grows the slowest: $f_{1}(n)$
\item These functions grow at the same asymptotic rate and faster than $f_{1}(n)$: $f_{2}(n), f_{3}(n), \ldots$
\item These functions grow at the same asymptotic rate, but faster than $f_{2}(n)$: $f_{k}(n)$.
\end{itemize}


\noindent \\ Also below is an example of an \texttt{align} block to help you organize your work.
\begin{align*}
\lim_{n \to \infty} \frac{n^{2}}{2^{n}} &= \lim_{n \to \infty} \frac{2n}{\ln(2) \cdot 2^{n}} \\
&= \lim_{n \to \infty} \frac{2}{(\ln(2))^{2} \cdot 2^{n}} \\
&= 0.
\end{align*}
\newpage
\begin{enumerate} [label=(\alph*)]
\subsection{Problem 12\ref{1a}}
    \item \label{1a} $ n^2+200n$,\qquad  $n+1000000$, \qquad $n^3-20n^2$,\qquad  $\frac{n^2}{\sqrt{n}}$.
    \begin{proof}[Answer]
Let us first define our values for further use:
\begin{center}
$f_1(n)$ = $ n^2+200n$ \\
$f_2(n)$ =  $n+1000000$ \\
$f_3(n)$ = $n^3-20n^2$ \\
$f_4(n)$ = $\frac{n^2}{\sqrt{n}}$ \\
\end{center}
We will first start by comparing $f_1(n)$ with $f_2(n)$ to find which function will grow at a larger rate using L'hopitals rule: \\
\begin{align*}
\lim_{n \to \infty} \frac{n^2+200n}{n+1000000} &= \lim_{n \to \infty} \frac{n^2+200n}{n+1000000} = \frac{\infty}{\infty} \qquad Indeterminate \\
&= \lim_{n \to \infty} \frac{2n+200}{1} \\
&= \infty.
\end{align*}
We can see from the above that this will be  $f_2(n) \leq O(f_1(n))$, which is representative that $f_1(n)$ will be an upper bound of $f_2(n)$. Thus, $f_1(n) \geq \Omega(f_2(n))$, which is the lower bound. \\

We will now compare $f_1(n)$ with $f_4(n)$ to find which function will grow at a larger rate using L'hopitals rule: \\
\begin{align*}
\lim_{n \to \infty} \frac{n^2+200n}{\frac{n^2}{\sqrt{n}}} &= \lim_{n \to \infty} \frac{n^2+200n}{n^{3/2}}  = \frac{\infty}{\infty} \qquad Indeterminate  \\
&= \lim_{n \to \infty} \frac{2n+200}{\frac{3}{2}n^{\frac{1}{2}}}  = \frac{\infty}{\infty} \qquad Indeterminate \\
&= \lim_{n \to \infty} \frac{2}{\frac{3}{4}n^{-\frac{1}{2}}}  \\
&= \lim_{n \to \infty} \frac{2n^{-\frac{1}{2}}}{\frac{3}{4}}  = \frac{\infty}{3}  \\
&= \infty.
\end{align*} 
We can see from the above that this will actually be $f_4(n) \leq O(f_1(n))$, which is representative that $f_1(n)$ will be an upper bound of $f_4(n)$. Thus, $f_1(n) \geq \Omega(f_4(n))$, which is the lower bound. \\

We will now compare $f_1(n)$ with $f_3(n)$ to find which function will grow at a larger rate using L'hopitals rule: \\
\begin{align*}
\lim_{n \to \infty} \frac{n^2+200n}{n^{3}-20n^{2}} &= \lim_{n \to \infty} \frac{n^2+200n}{n^{3}-20n^{2}}  = \frac{\infty}{\infty} \qquad Indeterminate  \\
&= \lim_{n \to \infty} \frac{2n+200}{3n^{2}-40n}  = \frac{\infty}{\infty} \qquad Indeterminate \\
&= \lim_{n \to \infty} \frac{2}{6n-40}  = \frac{2}{\infty}  \\
&= 0.
\end{align*} 
We can see from the above that $f_3(n)$ will actually grow faster than $f_1(n)$, such that $f_1(n) \leq O(f_3(n))$. This shows that $f_3(n)$ will be an upper bound of $f_1(n)$. Thus, $f_3(n) \geq \Omega(f_1(n))$, which is the lower bound. We can see that since $f_3(n)$ will grow faster than $f_1(n)$ and $f_1(n)$ grows faster than both $f_2(n)$ and $f_3(n)$, than we know that it will grow the fastest out of our set of functions. \\

We do not know whether $f_2(n)$ or $f_4(n)$ grows faster than each other so we will compare the two functions to each other using L'hopital's rule: \\
\begin{align*}
\lim_{n \to \infty} \frac{n+1000000}{\frac{n^2}{\sqrt{n}}} &= \lim_{n \to \infty} \frac{n+1000000}{n^{3/2}}  = \frac{\infty}{\infty} \qquad Indeterminate  \\
&= \lim_{n \to \infty} \frac{1}{\frac{3}{2}n^{\frac{1}{2}}}  = \frac{1}{\infty}\\
&= 0.
\end{align*} 
We can see from the above that $f_4(n)$ will grow faster than $f_2(n)$, such that $f_2(n) \leq O(f_4(n))$. This shows that $f_4(n)$ will be an upper bound of $f_2(n)$. Thus, $f_4(n) \geq \Omega(f_2(n))$, which is the lower bound. \\

Out of our set of functions we see that:
\begin{center}
The functions that grows the fastest: $f_3(n)$ \\
The function that grows the slowest: $f_2(n)$ 

\end{center}
\end{proof}

\newpage
\subsection{Problem 12\ref{1b}}
    \item \label{1b} $ (\log_7 n)^7 $, \qquad $10 \log_5 n$, \qquad $\log_7 (n^7)$, \qquad $n^{1/1000}$.
    \emph{Hint:} Recall change of logarithmic base formula $\log_a x = \log_b x\cdot\log_a b$
    \begin{proof}[Answer]
Let us first start by listing out the functions that we will be using:
\begin{center}
$f_1(n)$=$ (\log_7 n)^7 $ \\
$f_2(n)$=$10 \log_5 n$ \\
$f_3(n)$=$\log_7 (n^7)$ \\
$f_4(n)$=$n^{1/1000}$ \\
\end{center}

I will first compare the functions of $f_1(n)$ and $f_3(n)$ using L'hopital's rule:
\begin{align*}
\lim_{n \to \infty} \frac{(\log_7 n)^7}{\log_7 (n^{7})} &= \lim_{n \to \infty} \frac{\log_7 n)^7}{\log_7 (n^{7})}  = \frac{\infty}{\infty} \qquad Indeterminate  \\
&= \lim_{n \to \infty} \frac{\frac{7(\log_7 n)^6}{n\ln7}}{\frac{7}{n\ln7}}\\
&= \lim_{n \to \infty} \frac{7(\log_7 n)^6}{n\ln7} * \frac{n\ln7}{7} \\
&= \lim_{n \to \infty} \frac{7(\log_7 n)^6}{7} = \frac{\infty}{7}\\
&= \infty
\end{align*} 

We can see from the above that $f_1(n)$ will grow faster than $f_3(n)$, where $f_1(n)$ will be an upper bound of $f_3(n)$, such that $f_3(n) \leq O(f_1(n))$. Thus, $f_1(n) \geq \Omega(f_3(n))$, which is the lower bound. \\

I will now compare f $f_1(n)$ and $f_2(n)$ using L'hopital's rule:\\
I will first change the base of $f_2(n)$ into $\log_7$ using the log base formula: 
\begin{align*}
10\log_5n &= 10(\frac{\log_7n}{\log_7 5}) \\
&= (\frac{10}{\log_7 5})\log_7 n
\end{align*}
I will now compare the two functions after performing the log base formula: \\
\begin{align*}
\lim_{n \to \infty} \frac{(\log_7 n)^7}{10\log_5n} &= \lim_{n \to \infty} \frac{\log_7 n)^7}{(\frac{10}{\log_7 5})\log_7 n}  = \frac{\infty}{\infty} \qquad Indeterminate  \\
&= (\frac{10}{\log_7 5}) * \lim_{n \to \infty} \frac{(\log_7 n)^7}{\log_7 n}\\
&= (\frac{10}{\log_7 5}) * \lim_{n \to \infty} \frac{(7(\log_7 n)^6) * \frac{1}{n\ln7}}{\frac{1}{n\ln7}}\\
&= (\frac{10}{\log_7 5}) * \lim_{n \to \infty} (7(\log_7 n)^6) = (\frac{10}{\log_7 5})  * \infty\\
&= \infty
\end{align*} 
We can see from the above that $f_1(n)$ will grow faster than $f_2(n)$, where $f_1(n)$ will be an upper bound of $f_2(n)$, such that $f_2(n) \leq O(f_1(n))$. Thus, $f_1(n) \geq \Omega(f_2(n))$, which is the lower bound. \\

I will now compare the functions $f_1(n)$ with $f_4(n)$, using L'hopital's rule: \\
\begin{align*}
\lim_{n \to \infty} \frac{(\log_7 n)^7}{n^\frac{1}{1000}} &= \lim_{n \to \infty} \frac{\log_7 n)^7}{n^\frac{1}{1000}}  = \frac{\infty}{\infty} \qquad Indeterminate  \\
\end{align*}
I will now do the log base formula on $f_1(n)$: \\
\begin{align*}
(\log_7n)^{7} = (\frac{\ln n}{\ln 7})^{7}
\end{align*}
\\
\begin{align*}
&= \lim_{n \to \infty} \frac{(\frac{\ln n}{\ln 7})^{7}}{n^{\frac{1}{1000}}} \\ 
&= \frac{1}{(\ln 7)^{7}}\lim_{n \to \infty} \frac{(\ln n)^{7}}{n^{\frac{1}{1000}}} \\ 
&= \frac{1}{(\ln 7)^{7}}\lim_{n \to \infty} \frac{7(\ln n)^{6} * \frac{1}{n}}{\frac{1}{1000}n^{\frac{-999}{1000}}} \\
&= \frac{7000}{(\ln 7)^{7}}\lim_{n \to \infty} \frac{(\ln n)^{6}}{n^{\frac{1}{1000}}} \\
&= \frac{7000}{(\ln 7)^{7}}\lim_{n \to \infty} \frac{6(\ln n)^{5} * \frac{1}{n}}{\frac{1}{1000}n^{\frac{-999}{1000}}} \\
&= \frac{42,000,000}{(\ln 7)^{7}}\lim_{n \to \infty} \frac{(\ln n)^{5}}{n^{\frac{1}{1000}}} \\
&= \frac{42,000,000}{(\ln 7)^{7}}\lim_{n \to \infty} \frac{5(\ln n)^{4} * \frac{1}{n}}{\frac{1}{1000}n^{\frac{-999}{1000}}} \\
&= \frac{2.1*10^{11}}{(\ln 7)^{7}}\lim_{n \to \infty} \frac{(\ln n)^{4}}{n^{\frac{1}{1000}}} \\
&= \frac{2.1*10^{11}}{(\ln 7)^{7}}\lim_{n \to \infty} \frac{4(\ln n)^{3} * \frac{1}{n}}{\frac{1}{1000}n^{\frac{-999}{1000}}} \\
&= \frac{8.4*10^{14}}{(\ln 7)^{7}}\lim_{n \to \infty} \frac{(\ln n)^{3}}{n^{\frac{1}{1000}}} \\
&= \frac{8.4*10^{14}}{(\ln 7)^{7}}\lim_{n \to \infty} \frac{3(\ln n)^{2} * \frac{1}{n}}{\frac{1}{1000}n^{\frac{-999}{1000}}} \\
&= \frac{2.52*10^{18}}{(\ln 7)^{7}}\lim_{n \to \infty} \frac{(\ln n)^{2}}{n^{\frac{1}{1000}}} \\
&= \frac{2.52*10^{18}}{(\ln 7)^{7}}\lim_{n \to \infty} \frac{2(\ln n)^{1} * \frac{1}{n}}{\frac{1}{1000}n^{\frac{-999}{1000}}} \\
&= \frac{5.04*10^{21}}{(\ln 7)^{7}}\lim_{n \to \infty} \frac{(\ln n)^{1}}{n^{\frac{1}{1000}}} \\
&= \frac{5.04*10^{21}}{(\ln 7)^{7}}\lim_{n \to \infty} \frac{\frac{1}{n}}{\frac{1}{1000}n^{\frac{-999}{1000}}} \\
&= \frac{5.04*10^{24}}{(\ln 7)^{7}}\lim_{n \to \infty} \frac{\frac{1}{n}}{n^{\frac{-999}{1000}}} = \frac{0}{\infty} = 0\\
&= \frac{5.04*10^{24}}{(\ln 7)^{7}} * 0 \\
&= 0
\end{align*} 
From the above we can see that the function $f_1(n)$ will grow slower than the function $f_4(n)$, such that the function $f_4(n)$ will be an upper bound of $f_1(n)$. This is because as $f_4(n)$ is approaching infinity it will become an upper bound of $f_1(n)$.  Such that  $f_1(n) \leq O(f_3(n))$. Since $f_4(n)$ grows faster than the function $f_1(n)$, we can infer that it will grow the fastest amongst our set of functions. Thus, $f_4(n) \geq \Omega(f_1(n))$, which is the lower bound. \\

I will now compare the functions of $f_2(n)$ and $f_3(n)$ to see out of these functions which one will grow faster using L'hopital's rule: \\
I will first change the base of $f_2(n)$ into $\log_7$ using the log base formula: 
\begin{align*}
10\log_5n &= 10(\frac{\log_7n}{\log_7 5}) \\
&= (\frac{10}{\log_7 5})\log_7 n
\end{align*}
\begin{align*}
\lim_{n \to \infty} \frac{10\log_5n}{\log_7 (n^{7})} &= \lim_{n \to \infty} \frac{(\frac{10}{\log_7 5})\log_7 n}{\log_7 (n^{7})}  = \frac{\infty}{\infty} \qquad Indeterminate  \\
&= (\frac{10}{\log_7 5}) * \lim_{n \to \infty} \frac{\log_7 n}{\log_7 (n^{7})}\\
&= (\frac{10}{\log_7 5}) * \lim_{n \to \infty} \frac{\frac{1}{n\ln7}}{\frac{7}{n\ln7}}\\
&= (\frac{10}{\log_7 5}) * \lim_{n \to \infty} \frac{1}{n\ln7} * \frac{n\ln7}{7}\\
&= (\frac{10}{\log_7 5}) * \lim_{n \to \infty} \frac{1}{7}\\
&= (\frac{10}{\log_7 5}) * \frac{1}{7} \\
&= (\frac{10}{7(\log_7 5)}) 
\end{align*} 
We can see from the above that the functions $f_2(n)$ and $f_3(n)$ will grow at the same rate of each other since the limit is a constant. Such that it will be $f_2(n) \in \Theta(f_3(n))$. \\
    \end{proof}
\end{enumerate}

\end{required}

\newpage
\section{Standard 13- Asymptotics II (Calculus II techniques): }
\begin{required}



    {\itshape For each of the following questions, put the growth rates in order, from slowest-growing to fastest. That is, if your answer is $f_1(n), f_2(n), \dotsc, f_k(n)$, then $f_i(n) \leq O(f_{i+1}(n))$ for all $i$. If two adjacent ones are asymptotically the same (that is, $f_i(n) = \Theta(f_{i+1}(n))$), you must specify this as well. 
    Justify your answer (show your work). You may assume transitivity: if $f(n) \leq O(g(n))$ and $g(n) \leq O(h(n))$, then $f(n) \leq O(h(n))$, and similarly for little-oh, etc. The same instructions as for Problem 1 apply.}
    \begin{enumerate}[label=(\alph*)]
\subsection{Problem 13\ref{2a}}
        \item \label{2a} $n^{\log_5 n}$, \qquad $4$ ,\qquad $n^{\log_3 n}$,\qquad  $n^{\log_n(n^2)}$, \qquad $ n^{\log_n 5}$ .
        \begin{proof}[Answer]
I will first define the functions that we will be using:
\begin{center}
$f_1(n)$ = $n^{\log_5 n}$ \\
$f_2(n)$ = 4 \\
$f_3(n)$ = $n^{\log_3 n}$ \\
$f_4(n)$ = $n^{\log_n(n^2)}$ \\
$f_5(n)$ = $ n^{\log_n 5}$ \\
\end{center}

We will first compare functions $f_1(n)$ and $f_5(n)$ to see which function will grow faster: \\
\begin{align*}
&= \lim_{n \to \infty} \frac{n^{\log_5 n}}{n^{\log_n 5}} = \lim_{n \to \infty} \frac{n^{\log_5 n}}{n^{\log_n 5}}  = \frac{\infty}{\infty} \qquad Indeterminate  \\
&= \lim_{n \to \infty} \frac{n^{\log_5 n}}{n^{\log_n 5}}\\
&= \lim_{n \to \infty} \frac{n^{\log_5 n}}{5}\\
&= \lim_{n \to \infty} \frac{n^{\log_5 n}}{5}\\
&= \frac {\infty}{5} \\
&= \infty
\end{align*} 
We can see from the above that the function $f_1(n)$ will grow faster than the function $f_5(n)$. Such that $f_5(n) \leq O(f_1(n))$. Further, the function $f_1(n)$ will be an upper bound of $f_5(n)$. Thus, the function $f_1(n) \geq \Omega(f_5(n))$, such that it will be a lower bound. \\

I will now compare the functions of $f_2(n)$ and $f_5(n)$, to see which function will grow faster: \\
\begin{align*}
&= \lim_{n \to \infty} \frac{4}{n^{\log_n 5}} \\
&= \lim_{n \to \infty} \frac{4}{5} \\
&= \frac {4}{5}
\end{align*} 
We can see from the above that the functions of $f_2(n)$ and $f_5(n)$ will grow at the same rate as each other, such that they will be big theta of one another. Such that it will be $f_2(n) \in \Theta(f_5(n))$.

I will now compare the functions of $f_3(n)$ and $f_1(n)$ to see which function will grow faster: \\
I will change the base of function $f_3(n)$ using the log base change formula:
\begin{align*}
n^{\log_3 n} = \frac{\log_5 n}{\log_5 3}
\end{align*}
\begin{align*}
&= \lim_{n \to \infty} \frac{n^{\log_3 n}}{n^{\log_5 n}} = \lim_{n \to \infty} \frac{n^{\log_3 n}}{n^{\log_5 n}}  = \frac{\infty}{\infty} \qquad Indeterminate  \\
&= \lim_{n \to \infty} n^{{\log_3 n}-{\log_5 n}} \\
&= \lim_{n \to \infty} n^{{\frac{1}{\log_5 3}}-{\log_5 n}} \\
&= \lim_{n \to \infty} \frac{1}{\log_5 3} - 1 \ge 1 \\
&= \infty^{\infty} \\
&= \infty
\end{align*} 
We can see from the above that the function $f_3(n)$ will grow faster than the function of f $f_1(n)$, such that it will become $f_1(n) \leq O(f_3(n))$, which will become an upper bound. Further, the function $f_3(n) \leq \Omega(f_1(n))$.

I will now compare the functions of $f_3(n)$ and $f_4(n)$ to see which function will grow faster: \\
\begin{align*}
&= \lim_{n \to \infty} \frac{n^{\log_3 n}}{n^{\log_n(n^2)}} = \lim_{n \to \infty} \frac{n^{\log_3 n}}{n^{\log_n(n^2)}}  = \frac{\infty}{\infty} \qquad Indeterminate  \\
&= \lim_{n \to \infty} \frac{n^{\log_3 n}}{n^2} \\
&= \lim_{n \to \infty} n^{{\log_3 n}-{2}} \\
&= \lim_{n \to \infty} n^{{\log_3 n}-{\log_3 9}} \\
&= \lim_{n \to \infty} n^{\log_3 \frac{n}{9}}\\
&= \infty^{\infty}\\
&= \infty
\end{align*} 
From the above we can see that the function $f_3(n)$ will grow faster than the function $f_4(n)$, such that $f_4(n) \leq O(f_3(n))$. Where the function $f_3(n)$ will be an upper bound of $f_4(n)$. Such that $f_3(n) \leq \Omega(f_4(n))$.\\

I will now compare the function of $f_1(n)$ and  $f_4(n)$ to see which function will grow faster: \\
\begin{align*}
&= \lim_{n \to \infty} \frac{n^{\log_5 n}}{n^{\log_n(n^2)}} = \lim_{n \to \infty} \frac{n^{\log_5 n}}{n^{\log_n(n^2)}}  = \frac{\infty}{\infty} \qquad Indeterminate  \\
&= \lim_{n \to \infty} \frac{n^{\log_5 n}}{n^2} \\
&= \lim_{n \to \infty} n^{{\log_5 n}-{2}} \\
&= \lim_{n \to \infty} n^{{\log_5 n}-{\log_5 25}} \\
&= \lim_{n \to \infty} n^{\log_5 \frac{n}{25}}\\
&= \infty^{\infty}\\
&= \infty
\end{align*} 
We can see from the above that the function of $f_1(n)$ will grow faster than the function of $f_4(n)$, such that $f_4(n) \leq O(f_1(n))$, where the function $f_1(n)$ will be an upper bound of the function $f_4(n)$. Further, this will cause for the function $f_4(n)$, to be a lower bound of the function $f_1(n)$, such that $f_1(n) \leq \Omega(f_4(n))$.


        \end{proof}
        \newpage

\subsection{Problem 13\ref{2b}}
        \item \label{2b} $n!$ , \qquad $2^n$ ,\qquad  $2^{n/3}$, \qquad  $n^n$,\qquad $2^{n-2}$,\qquad  $\sqrt{n^{2n+1}}$. (\emph{Hint:} Recall \href{https://en.wikipedia.org/wiki/Stirling\%27s_approximation}{Stirling's approximation}, which says that $n! \sim \left(\frac{n}{e}\right)^n \sqrt{2 \pi n}$, i.e. $\lim_{n \to \infty} \frac{n!}{\left(\frac{n}{e}\right)^n \sqrt{2 \pi n}} = 1$.).
        \begin{proof}[Answer]

        \end{proof}
\end{enumerate}

\end{required}

\newpage
\section{Standard 14- Analyzing Code I: (Independent nested loops)}
\begin{required}


Analyze the \textit{worst-case} runtime of the following algorithms. Clearly derive the runtime complexity function $T(n)$ for this algorithm, and then find a tight asymptotic bound for $T(n)$ (that is, find a function $f(n)$ such that $T(n) = \Theta(f(n))$). Avoid heuristic arguments from 2270/2824 such as multiplying the complexities of nested loops.



\begin{algorithm}
\caption{Nested Algorithm 1}\label{alg:Nested2}
\begin{algorithmic}[1]
\Procedure{Foo2}{$\text{Integer } n$}
\For{$i \gets 1; i \leq n; i \gets i+2 $}
	\For{$j \gets 1; j \leq n; j \gets 3*j$}
		\State \textbf{print} \text{``Hi"}
	\EndFor
\EndFor
\EndProcedure
\end{algorithmic}
\end{algorithm}
\begin{proof}[Answer]
\end{proof}
\end{required}

\newpage
\section{Standard 15- Analyzing Code II: (Dependent nested loops)}
\begin{required}


Analyze the \textit{worst-case} runtime of the following algorithms. Clearly derive the runtime complexity function $T(n)$ for this algorithm, and then find a tight asymptotic bound for $T(n)$ (that is, find a function $f(n)$ such that $T(n) = \Theta(f(n))$). Avoid heuristic arguments from 2270/2824 such as multiplying the complexities of nested loops.


\begin{algorithm}
\caption{Nested Algorithm 2}\label{alg:NestedDependent1}
\begin{algorithmic}[1]
\Procedure{Boo1}{$\text{Integer } n$}
\For{$i \gets 1; i \leq n; i \gets i+1$}
	\For{$j \gets i; j \leq n; j \gets j + 1$}
		\State \textbf{print} \text{``Hi"}
	\EndFor
\EndFor
\EndProcedure
\end{algorithmic}
\end{algorithm}
\end{required}

\begin{proof}[Answer]
\end{proof}

\end{document} % NOTHING AFTER THIS LINE IS PART OF THE DOCUMENT