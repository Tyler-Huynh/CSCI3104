\documentclass[11pt]{article} 
\usepackage[english]{babel}
\usepackage[utf8]{inputenc}
\usepackage[margin=0.5in]{geometry}
\usepackage{amsmath}
\usepackage{amsthm}
\usepackage{amsfonts}
\usepackage{amssymb}
\usepackage[usenames,dvipsnames]{xcolor}
\usepackage{graphicx}
\usepackage[siunitx]{circuitikz}
\usepackage{tikz}
\usepackage[colorinlistoftodos, color=orange!50]{todonotes}
\usepackage{hyperref}
\usepackage[numbers, square]{natbib}
\usepackage{fancybox}
\usepackage{epsfig}
\usepackage{soul}
\usepackage[framemethod=tikz]{mdframed}
\usepackage[shortlabels]{enumitem}
\usepackage[version=4]{mhchem}
\usepackage{multicol}

\usepackage{mathtools}
\usepackage{comment}
\usepackage{enumitem}
\usepackage[utf8]{inputenc}
\usepackage[linesnumbered,ruled,vlined]{algorithm2e}
\usepackage{listings}
\usepackage{color}
\usepackage[numbers]{natbib}
\usepackage{subfiles}
\usepackage{tkz-berge}


\newtheorem{prop}{Proposition}[section]
\newtheorem{thm}{Theorem}[section]
\newtheorem{lemma}{Lemma}[section]
\newtheorem{cor}{Corollary}[prop]

\theoremstyle{definition}
\newtheorem{definition}{Definition}

\theoremstyle{definition}
\newtheorem{required}{Problem}
\newtheorem*{requiredHC}{Problem HC}

\theoremstyle{definition}
\newtheorem{ex}{Example}


\setlength{\marginparwidth}{3.4cm}
%#########################################################

%To use symbols for footnotes
\renewcommand*{\thefootnote}{\fnsymbol{footnote}}
%To change footnotes back to numbers uncomment the following line
%\renewcommand*{\thefootnote}{\arabic{footnote}}

% Enable this command to adjust line spacing for inline math equations.
% \everymath{\displaystyle}

% _______ _____ _______ _      ______ 
%|__   __|_   _|__   __| |    |  ____|
%   | |    | |    | |  | |    | |__   
%   | |    | |    | |  | |    |  __|  
%   | |   _| |_   | |  | |____| |____ 
%   |_|  |_____|  |_|  |______|______|
%%%%%%%%%%%%%%%%%%%%%%%%%%%%%%%%%%%%%%%

\title{
\normalfont \normalsize 
\textsc{CSCI 3104 Fall 2022 \\ 
Instructors: Prof. Grochow and Chandra Kanth Nagesh} \\
[10pt] 
\rule{\linewidth}{0.5pt} \\[6pt] 
\huge Quiz 4 S10 \\
\rule{\linewidth}{2pt}  \\[10pt]
}
%\author{Your Name}
\date{}

\begin{document}
\definecolor {processblue}{cmyk}{0.96,0,0,0}
\definecolor{processred}{rgb}{200, 0, 0}
\definecolor{processgreen}{rgb}{0, 255, 0}
\DeclareGraphicsExtensions{.png}
\DeclareGraphicsExtensions{.gif}
\DeclareGraphicsExtensions{.jpg}

\maketitle


%%%%%%%%%%%%%%%%%%%%%%%%%
%%%%%%%%%%%%%%%%%%%%%%%%%%
%%%%%%%%%%FILL IN YOUR NAME%%%%%%%
%%%%%%%%%%AND STUDENT ID%%%%%%%%
%%%%%%%%%%%%%%%%%%%%%%%%%%
\noindent
Due Date \dotfill Thursday Oct 13, 2022 8pm MT \\
Name \dotfill \textbf{Tyler Huynh} \\
Student ID \dotfill \textbf{109603994} \\
Quiz Code (enter in Canvas to get access to the LaTeX template) \dotfill \textbf{TYHHH}


\tableofcontents

\section*{Instructions}
\addcontentsline{toc}{section}{Instructions}
 \begin{itemize}
	\item You may either type your work using this template, or you may handwrite your work and embed it as an image in this template. \textbf{If you choose to handwrite your work, the image must be legible, and oriented so that we do not have to rotate our screens to grade your work.} We have included some helpful LaTeX commands for including and rotating images commented out near the end of the LaTeX template.
	\item You should submit your work through the \textbf{class Gradescope page} only. Please submit one PDF file, compiled using this \LaTeX \ template.
	\item You may not need a full page for your solutions; pagebreaks are there to help Gradescope automatically find where each problem is. Even if you do not attempt every problem, please submit this document with no fewer pages than the blank template (or Gradescope has issues with it).

	\item You \textbf{may not collaborate with other students}. \textbf{Copying from any source is an Honor Code violation. Furthermore, all submissions must be in your own words and reflect your understanding of the material.} If there is any confusion about this policy, it is your responsibility to clarify before the due date. 

	\item Posting to \textbf{any} service including, but not limited to Chegg, Discord, Reddit, StackExchange, etc., for help on an assignment is a violation of the Honor Code.

	\item You \textbf{must} virtually sign the Honor Code (see Section \ref{HonorCode}). Failure to do so will result in your assignment not being graded.
\end{itemize}


\newpage
\section*{Honor Code (Make Sure to Virtually Sign)} \label{HonorCode}
\addcontentsline{toc}{section}{Honor Code (Make Sure to Virtually Sign)}
\hypertarget{HonorCode}{}

\begin{requiredHC}
\begin{itemize}
\item My submission is in my own words and reflects my understanding of the material.
\item Any collaborations and external sources have been clearly cited in this document.
\item I have not posted to external services including, but not limited to Chegg, Reddit, StackExchange, etc.
\item I have neither copied nor provided others solutions they can copy.
\end{itemize}

%\noindent In the specified region below, clearly indicate that you have upheld the Honor Code. Then type your name. 
\end{requiredHC}

\begin{proof}[I agree to the above, Tyler Huynh.]
%% Typing "I agree to the above," followed by your name is sufficient.
\end{proof}


\newpage
\setcounter{section}{9}
\section{Standard 10: Network flow terminology}

\setcounter{required}{9}
\begin{required} 
Consider the following flow network, with the following flow configuration $f$ as indicated below. 
\begin{center}
	\begin {tikzpicture}[-latex ,auto ,node distance =2 cm and 3cm ,on grid ,
	semithick ,
	state/.style ={ circle ,top color =white , bottom color = processblue!20 ,
	draw,processblue , text=blue , minimum width =1 cm}]

	\node[state] (A) {$s$};
	\node[state] (B) [above right = of A] {$B$};
	\node[state] (C) [below right = of A] {$C$};
	\node[state] (D) [right = of B] {$D$};
	\node[state] (E) [right = of C] {$E$};
	\node[state] (t) [below right = of D] {$t$};
	
	\path (A) edge node[below] {$3 / 5$} (B);
	\path (A) edge node[above] {$2 / 6$} (C);
	\path (B) edge node[left] {$1 / 1$} (C);
	\path (B) edge node[above] {$3 / 4$} (D);
	\path (E) edge node[right] {$1 /  5$} (B);
	\path (C) edge node[above] {$2 / 7$} (E);
	\path (D) edge node[right] {$3 / 3$} (E);
	\path (D) edge node[below] {$0 / 3$} (t);
	\path (E) edge node[above] {$4 / 6$} (t);
	
	\end{tikzpicture}  
\end{center}

There are four parts to this question, (a)--(d); be sure to answer all four!

\end{required}

\renewcommand{\theenumi}{\alph{enumi}}
\begin{enumerate}
\item There is one vertex at which the above configuration is \emph{not} a flow. Specify the vertex at which it is not a flow, and argue why it is not a flow there---say what you are calculating and show your calculation.

\begin{proof}[Answer]
The vertex that would not be a flow would be the vertex C because there is 3 flow going into this vertex, but there is only 2 flow that is going out of this vertex. On the edge from $S \to C$ has 2 flow. On the edge from $B \to C$ has 1 flow. But on the edge from $C \to E$ there is only 2 flow that exists. These flows do not equal each other, thus it is not a valid flow for the configuration of $f$.
\end{proof}

\vfill
\item Consider the path $s \to B \to D \to t$. Is it an augmenting path? If so, specify how much additional flow can be pushed along this path, and show your calculation. If not, explain why not.

\begin{proof}[Answer]
On the path $s \to B \to D \to t$ it is an augmenting path. On the edge from $s \to B$ we can push an additional flow of 2. On the edge from $B \to D$ we can push an additional flow of 1. On the edge from $D \to t$ we can push a total of 3 flow. On the path since the maximum capacity of the edge from $B \to D$ is 4 and we are already pushing 3 flow on this edge than for the entire path we can only push an additional 1 flow, thus our total additional flow would be 1.
\end{proof}

\vfill
\newpage 

\item Consider the path $s \to B \to E \to t$. Is it an augmenting path? If so, specify how much additional flow can be pushed along this path, and show your calculation. If not, explain why not.

\begin{proof}[Answer]
The path of $s \to B \to E \to t$ is a flow augmenting path because there exists a valid flow on this path. On the edge from $s \to B$ we can push an additional flow of 2 on this edge. On the edge from $B \to E$ the maximum amount of flow that we can push back on this edge would only be 1 as we are only receiving 1 flow on this edge. On the edge from $E \to t$ we can push an additional flow of 2. However, since we can only push maximum flow of 1 on the edge $B \to E$ the additional amount of flow we can push is 1, thus the total additional flow that we can push on this path is 1.
\end{proof}

\vfill
\item Consider the cut $\{(s,B), (s,C)\}$ (in terms of sets of vertices, this is the cut that divides $\{s\}$ from $\{B,C,D,E,t\}$). What is the capacity of this cut? Explain your answer / show your calculation.

\begin{proof}[Answer]
The cut on the edges of $\{(s,B), (s,C)\}$ would have a capacity of 11. The capacities of the cut are the capacities of the edges that are crossing the cut such that, 5 + 6 = 11. Thus, the capacity of this cut is 11.
\end{proof}
\vfill

\end{enumerate}


%%%%%%%%%%%%%%%%%%%%%%%%%%%%%%%%%%%%%%%%%%%%%%%%%%
\end{document} % NOTHING AFTER THIS LINE IS PART OF THE DOCUMENT



