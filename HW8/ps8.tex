\documentclass[11pt]{article} 
\usepackage[english]{babel}
\usepackage[utf8]{inputenc}
\usepackage[margin=0.5in,top=0.5in,bottom=0.75in]{geometry}
\usepackage{amsmath}
\usepackage{amsthm}
\usepackage{amsfonts}
\usepackage{amssymb}
\usepackage[usenames,dvipsnames]{xcolor}
\usepackage{graphicx}
\usepackage[siunitx]{circuitikz}
\usepackage{tikz}
\usepackage[colorinlistoftodos, color=orange!50]{todonotes}
\usepackage{hyperref}
\usepackage[numbers, square]{natbib}
\usepackage{fancybox}
\usepackage{epsfig}
\usepackage{soul}
\usepackage[framemethod=tikz]{mdframed}
\usetikzlibrary{positioning, automata, backgrounds}
\usepackage{tikz}\usetikzlibrary{arrows.meta,backgrounds,calc,quotes}
\usepackage[shortlabels]{enumitem}
\usepackage[version=4]{mhchem}
\usepackage{multicol}
\usepackage{forest}
\usepackage{mathtools}
\usepackage{comment}
\usepackage{enumitem}
\usepackage[utf8]{inputenc}
\usepackage[linesnumbered,ruled,vlined]{algorithm2e}
\usepackage{listings}
\usepackage{color}
\usepackage[numbers]{natbib}
\usepackage{subfiles}
\usepackage{tkz-berge}
\usepackage{mathtools}
\DeclarePairedDelimiter\ceil{\lceil}{\rceil}
\DeclarePairedDelimiter\floor{\lfloor}{\rfloor}



\newtheorem{prop}{Proposition}[section]
\newtheorem{thm}{Theorem}[section]
\newtheorem{lemma}{Lemma}[section]
\newtheorem{cor}{Corollary}[prop]

\theoremstyle{definition}
\newtheorem{definition}{Definition}

\theoremstyle{definition}
\newtheorem{required}{Problem}
\newtheorem*{requiredHC}{Problem HC}


\theoremstyle{definition}
\newtheorem{ex}{Example}

\newcommand{\interval}[4]{\draw (#2, #1) -- (#3, #1); % Usage: \interval{height}{start}{end}{label}
\draw (#2, #1-0.11) -- (#2, #1+0.11); % draw left whisker
\draw (#3, #1-0.11) -- (#3, #1+0.11); % draw right whisker
\node[] at (#2-0.25, #1) {#4};
}

\tikzset{>={Stealth[length=7pt]}}
\tikzset{
    vertex/.style={circle,draw,minimum size=16,inner sep=0pt,font=\normalsize},
    edgelabel/.style={rectangle,draw=none,font=\footnotesize,outer sep=0pt},
    every node/.style={vertex},
    every edge quotes/.append style={edgelabel},
    every to/.append style={every node/.style={edgelabel}},
    wide/.style={line width=4pt,>={Stealth[length=18pt]}},
    directed/.style={arrows={->},font=\small},
    caption/.style={text width=6cm,align=center,rectangle,draw},
}


\setlength{\marginparwidth}{3.4cm}
%#########################################################

%To use symbols for footnotes
\renewcommand*{\thefootnote}{\fnsymbol{footnote}}
%To change footnotes back to numbers uncomment the following line
%\renewcommand*{\thefootnote}{\arabic{footnote}}

% Enable this command to adjust line spacing for inline math equations.
% \everymath{\displaystyle}

% _______ _____ _______ _      ______ 
%|__   __|_   _|__   __| |    |  ____|
%   | |    | |    | |  | |    | |__   
%   | |    | |    | |  | |    |  __|  
%   | |   _| |_   | |  | |____| |____ 
%   |_|  |_____|  |_|  |______|______|
%%%%%%%%%%%%%%%%%%%%%%%%%%%%%%%%%%%%%%%

\title{
\normalfont \normalsize 
\textsc{CSCI 3104 Fall 2022 \\ 
Instructors: Prof. Grochow and Chandra Kanth Nagesh} \\
[10pt] 
\rule{\linewidth}{0.5pt} \\[6pt] 
\huge Problem Set 8 \\
\rule{\linewidth}{2pt}  \\[10pt]
}
%\author{}
\date{}

\begin{document}

\definecolor{processblue}{cmyk}{0.96,0,0,0}
\definecolor{processred}{rgb}{200, 0, 0}
\definecolor{processgreen}{rgb}{0, 255, 0}
\DeclareGraphicsExtensions{.png}
\DeclareGraphicsExtensions{.gif}
\DeclareGraphicsExtensions{.jpg}

\maketitle


%%%%%%%%%%%%%%%%%%%%%%%%%
%%%%%%%%%%%%%%%%%%%%%%%%%%
%%%%%%%%%%FILL IN YOUR NAME%%%%%%%
%%%%%%%%%%AND STUDENT ID%%%%%%%%
%%%%%%%%%%%%%%%%%%%%%%%%%%
\noindent
Due Date \dotfill November 1, 2022 \\
Name \dotfill \textbf{Tyler Huynh} \\
Student ID \dotfill \textbf{109603994} \\
Collaborators \dotfill \textbf{N/A}

\tableofcontents

\section*{Instructions}
\addcontentsline{toc}{section}{Instructions}
 \begin{itemize}
	\item The solutions \textbf{must be typed}, using proper mathematical notation. We cannot accept hand-written solutions. Useful links and references on \LaTeX can be found \href{https://canvas.colorado.edu/courses/75824/pages/latex}{here on Canvas}.
	\item You should submit your work through the \textbf{class Canvas page} only. Please submit one PDF file, compiled using this \LaTeX \ template.
	\item You may not need a full page for your solutions; pagebreaks are there to help Gradescope automatically find where each problem is. Even if you do not attempt every problem, please submit this document with no fewer pages than the blank template (or Gradescope has issues with it).

	\item You are welcome and encouraged to collaborate with your classmates, as well as consult outside resources. You must \textbf{cite your sources in this document.} \textbf{Copying from any source is an Honor Code violation. Furthermore, all submissions must be in your own words and reflect your understanding of the material.} If there is any confusion about this policy, it is your responsibility to clarify before the due date. 

	\item Posting to \textbf{any} service including, but not limited to Chegg, Reddit, StackExchange, etc., for help on an assignment is a violation of the Honor Code.

	\item You \textbf{must} virtually sign the Honor Code (see Section \hyperlink{HonorCode}{Honor Code}). Failure to do so will result in your assignment not being graded.
\end{itemize}


\section*{Honor Code (Make Sure to Virtually Sign the Honor Pledge)} 
\addcontentsline{toc}{section}{Honor Code (Make Sure to Virtually Sign the Honor Pledge)}
\hypertarget{HonorCode}{}

\begin{requiredHC}
On my honor, my submission reflects the following:
\begin{itemize}
\item My submission is in my own words and reflects my understanding of the material.
\item Any collaborations and external sources have been clearly cited in this document.
\item I have not posted to external services including, but not limited to Chegg, Reddit, StackExchange, etc.
\item I have neither copied nor provided others solutions they can copy.
\end{itemize}

\noindent In the specified region below, clearly indicate that you have upheld the Honor Code. Then type your name. 
\end{requiredHC}

\begin{proof}[Honor Pledge]
I, \textbf{Tyler Huynh} on my honor pledge that my submission is a reflection of my own understanding of the material, any and all collaborations/sources have been properly cited, I have not posted any material to external sources, and I have not copied other solutions as my own.\end{proof}


\newpage
\setcounter{section}{20}
\section{Standard 21 -- Dynamic Programming: Identify precise subproblems}

\setcounter{required}{20}
\begin{required}
The goal of this standard is to practice identifying the recursive structure. To be clear, you are \textbf{not} being asked for a precise mathematical recurrence. Rather, you are being asked to clearly and precisely identify the cases to consider. Identifying the cases can sometimes provide enough information to design a dynamic programming solution.


\subsection*{Problem 21(a)}
\addcontentsline{toc}{subsection}{Problem 21(a)}

\renewcommand{\theenumi}{\alph{enumi}}
\begin{enumerate}
\item Consider the \textsf{Stair Climbing} problem, defined as follows.
\begin{itemize}
\item \textsf{Instance:} Suppose we have $n$ stairs, labeled $s_{1}, \ldots, s_{n}$. Associated with each stair $s_{k}$ is a number $a_{k} \geq 1$. At stair $s_{k}$, we may jump forward $i$ stairs, where $i \in \{ 1, 2, \ldots, a_{k}\}$. You start on $s_{1}$.

\item \textsf{Solution:} The number of ways to to reach $s_{n}$ from $s_{1}$.
\end{itemize}

\noindent \\ \textbf{Your job} is to clearly identify the recursive structure. That is, suppose we are solving the subproblem at stair $s_{k}$. What precise sub-problems do we need to consider?

\begin{proof}[Answer] \textbf{Referenced Levet Notes} \\
I will begin by providing an example: \\ 

We are currently on stair $s_k$ where we can jump forward i stairs. Our precise subproblem will consist of when we are at stair $s_k$ how many ways it will take to get from the next step of $s_{k+i}$ ,where $i \in \{ 1, 2, \ldots, a_{k}\}$, to the  step of $s_n$, the last step. \\

Below I will show an arbitrary example: \\
Suppose we are at stair $s_k$,  $T(k)$ represents the number of ways to reach $s_n$ from $s_k$, such that: \\
\begin{center}
When we recurse on $T(k)$ we need to consider the case of $T(k+i)$ for all $i \in \{ 1, 2, \ldots, a_{k}\}$, in order to determine how many steps it will take us to get to $s_n$ from $s_{k+i}$. This will represent our precise subproblem for this problem\\
\end{center}
From this we can see that when we are at the base case of $s_n$ there will only be one way to get to $s_n$ from $s_n$. \\

From the above we can see that our precise subproblem for this problem would be how many ways it will take us to reach $s_n$ from $s_{k+i}$ for all $i \in \{ 1, 2, \ldots, a_{k}\}$ \\

Our base case for this problem would be $s_n$ where we know that there only exists one way to get to $s_n$ from $s_n$\\

Our recursive structure for $T(k)$ would be to find $T(k+i)$ for all $i \in \{ 1, 2, \ldots, a_{k}\}$.
\end{proof}

\newpage
\subsection*{Problem 21(b)}
\addcontentsline{toc}{subsection}{Problem 21(b)}

\item Fix $n \in \mathbb{N}$. The \textit{Trust Game} on $n$ rounds is a two-player dynamic game. Here, Player I starts with \$100. The game proceeds as follows.
\begin{itemize}
\item \textbf{Round 1:} Player I takes a fraction of the \$100 (which could be nothing) to give to Player II. The money Player I gives to Player II is multiplied by 1.5 before Player II receives it. Player I keeps the remainder. (So for example, if Player I gives \$20 to Player II, then Player II receives \$30 and Player I is left with \$80).

\item \textbf{Round 2:} Player II can choose a fraction of the money they received to offer to Player I. The money offered to Player I increases by a multiple of $1.5$  before Player I receives it. Player II keeps the remainder.
\end{itemize}

\noindent More generally, at round $i$, the Player at the current round (Player I if $i$ is odd, and Player II if $i$ is even) takes a fraction of the money in the current pile to send to the other Player and keeps the rest. That money increases by a factor of $1.5$ before the other player receives it. The game terminates if the current player does not send any money to the other player, or if round $n$ is reached. At round $n$, the money in the pile is split evenly between the two players. \\

\noindent Each individual player wishes to maximize the total amount of money they receive. \\

\noindent \textbf{Your job} is to clearly identify the recursive structure. That is, at round $i$, what precise sub-problems does the current player need to consider? [\textbf{Hint:} Do we have a smaller instance of the Trust Game after each round?]

\begin{proof}[Answer]
To begin I will provide an example: \\

Based off of this question we can see that the base cases for the question would be: \\
\begin{center}
If either player were to not send any money to each other than the game would end. \\
Or, if we have reached the maximum amount of rounds of n.\\
\end{center}
For some arbitrary round $i$ the current player could send no money to the other player, such that it causes the game to end, otherwise if the player sends an amount of money that $\neq$ 0, than the game would continue, this would represent our \textbf{recursive structure.} \\

We then have a smaller instance of our trust game that is of size $(n-1)$ rounds. This smaller instance is our \textbf{precise subproblem.} \\

From this eventually we will reach a game of size 1, where only 1 round happened, which would cause the game to terminate. \\

\end{proof}
\end{enumerate}
\end{required}

\newpage
\section{Standard 22 -- Dynamic Programming: Write Down Recurrences}

\subsection*{Problem 22(a)}
\addcontentsline{toc}{subsection}{Problem 22(a)}

Suppose we have an $m$-letter alphabet $\Sigma = \{0, 1, \ldots, m-1\}$. Let $W_{n}$ be the set of strings $\omega \in \Sigma^{n}$ such that $\omega$ does not have $00$ as a substring. Let $f_{n} := |W_{n}|$. Write down an \textbf{explicit recurrence for $f_{n}$, including the base cases.} Clearly justify each recursive term.
    
\begin{proof}[Answer] \textbf{Referenced Levet notes} \\
For this problem I will first start by identifying the base cases for the question such that, \\
\begin{center}
$n=0$, this will occur when the string is of size 0 where there only exists an empty string. \\

$n=1$, this will occur when the string is of size 1 with no possibility that the substring of $00$ will appear, where $m$ will represent the total size of $W_n$ \\

$n=2$, this will occur when the string is of size 2 with the possibility that the substring of $00$ will appear or not appear, where $m^2$ will represent the total size of $W_n$ \\
\end{center} 

I will now identify my explicit recurrence for $f_n$: \\

When $n \geq 2$ than there exists two possible cases: \\
\begin{itemize}
\item \textbf{Case 1:} \\
Consider if the substring that we are given to begin with a 0. Where $w_0 = 0$, if the substring were to begin with a 0 than the character after it must be in the set of $\Sigma^1= \{ 1, \ldots, m-1\}$. Thus we know that the second character in the substring must have $m-1$ options. This is to ensure that we do not have the substring of $00$ within our string. \\

For the rest of the characters in the string, it can be any string that does not contain $00$ of size $n-2$ this will be the same as $W_{n-2}$, we know the size of $W_{n-2}$ will be equal to the size of $f_{n-2}$, our recursive structure for this case will be: \\
\begin{center}
$f_n = (m-1)(f_{n-2})$
\end{center}
\item \textbf{Case 2:}\\
Consider if the substring that we are given were to begin with a character that is not 0. There are $m-1$ characters that are not 0. For the rest of our string it can be any string of size $n-1$ where the substring of $00$ does not exist where that will be $W_{n-1}$. Thus, the size of $W_{n-1}$ is equal to the size of $f_{n-1}$. 

From this the recurrence structure for this case will be: \\
\begin{center}
$f_n = (m-1)(f_{n-1})$
\end{center}
\end{itemize}

From the above two cases we can then compile our explicit recurrence for $f_n$ considering the two cases, such that: \\
\begin{center}
$f_n = (m-1)(f_{n-2}) + (m-1)(f_{n-1})$
\end{center}
Then our final explicit recurrence and including base cases: \\
\begin{align*}
f_n = \begin{cases}
1 & : n = 0, \\
m & : n = 1, \\
m^2 - 1 & : n = 2, \\
(m-1)(f_{n-2}) + (m-1)(f_{n-1}) & : n > 2.
\end{cases}
\end{align*}
\end{proof}

\newpage
\subsection*{Problem 22(b)}
\addcontentsline{toc}{subsection}{Problem 22(b)}

Suppose we have the alphabet $\Sigma = \{x, y\}$. For $n \geq 0$, let $W_{n}$ be the set of strings $\omega \in \{x, y\}^{n}$ where $\omega$ contains $yyy$ as a substring. Let $f_{n} := |W_{n}|$. Write down an explicit recurrence for $f_{n}$, including the base cases. Clearly justify each recursive term.

\begin{proof}[Answer]
I will first begin by identifying the base cases for the problem such that: \\
\begin{center}
The first base case will be when n = 0. In this case there are no strings that contain the substring $yyy$ because they are not large enough.\\
The second base case will be when n = 1. In this case there are no strings that contain the substring $yyy$ because they are not large enough.\\
The third base case will be when n=2. In this case there are no strings that contain the substring $yyy$ because they are not large enough.
\end{center}
Let $W_n$ be the set of string of length n which do contain the substring $yyy$, such that we have the following cases: \\
\begin{itemize}
\item \textbf{Case 1:} \\
We have that $w_0 \neq y$, from this we can see that if the first index in the substring were to not equal y then it must be $x$, which is just one option.

For the rest of the characters in the string, it can be any string that does contain $yyy$ of size $n-1$. This will be the same as $W_{n-1}$, we know the size of $W_{n-1}$ will be equal to the size of $f_{n-1}$, our recursive structure for this case will be: \\
\begin{center}
$f_n = 1 * f_{n-1}$
\end{center}
\item \textbf{Case 2a:} \\
We have that when $w_0 = y$ and $w_1 \neq y$, from this we can see that if the second index in the substring were to not equal y or equal y than the characters after it must be within the set of string $\omega \in \{x, y\}^{n}$. 

For the rest of the characters in the string, it can be any string that does contain the substring $yyy$ of size $n-2$. This will be the same as $W_{n-2}$, we know the size of $W_{n-2}$ will be equal to the size of $f_{n-2}$, our recursive structure for this case will be: \\
\begin{center}
$f_n = 1 * 1 * f_{n-2}$
\end{center}
\item \textbf{Case 2b:} \\
We have that when $w_0 = y$ and $w_1 = y$, from this we can see that if the second index in the substring were to not equal y or equal y than the characters after it must be within the set of string $\omega \in \{x, y\}^{n}$. 

For the rest of the characters in the string, it can be any string that does contain the substring $yyy$ of size $n-2$. This will be the same as $W_{n-2}$, we know the size of $W_{n-2}$ will be equal to the size of $f_{n-2}$, our recursive structure for this case will be: \\
\begin{center}
$f_n = 1 * 1 * f_{n-2}$
\end{center}
\end{itemize}
\end{proof}

\newpage
\section{Standard 23 -- Dynamic Programming: Using Recurrences to Solve}

\subsection*{Problem 23(a)}
\addcontentsline{toc}{subsection}{Problem 23(a)}

Given the following directed acyclic graph. Use dynamic programming to fill in a \textbf{one-dimensional} lookup table that counts number of paths from each node $j$ to 14, for $j \geq 1$. Note that a single vertex is considered a path of length $0$. \textbf{Fill in the lookup table for all vertices 1-14; and in addition, clearly show work for vertices 9-14}.

\begin{center}
	\begin {tikzpicture}[-latex ,auto ,node distance =2 cm and 3cm ,on grid ,
	semithick ,
	state/.style ={ circle ,top color =white , bottom color = processblue!20 ,
	draw,processblue , text=blue , minimum width =1 cm}]

	\node[state] (1) {$1$};
	\node[state] (2) [right = of 1] {$2$};
	\node[state] (3) [right = of 2] {$3$};
	\node[state] (4) [right = of 3] {$4$};
	\node[state] (5) [right = of 4] {$5$};
	\node[state] (6) [right = of 5] {$6$};
	
	\node[state] (7) [below = of 2] {7};
	\node[state] (8) [right = of 7] {8};	
	\node[state] (9) [right = of 8] {9};	
	\node[state] (10) [right = of 9] {10};	
	\node[state] (11) [right = of 10] {11};	
	
	\node[state] (12) [below = of 9] {12};
	\node[state] (13) [right = of 12] {13};
	\node[state] (14) [right = of 13] {14};
	
	\path (1) edge (2);
	\path (1) edge (7);
	\path (2) edge (3);
	\path (2) edge (7);
	\path (2) edge (8);
	\path (3) edge (4);
	\path (3) edge (8);
	\path (4) edge (5);
	\path (4) edge (9);
	\path (4) edge (10);
	\path (5) edge (6);
	\path (5) edge (11);
	\path (6) edge (11);
	\path (7) edge (8);
	\path (8) edge (9);
	\path (8) edge (12);
	\path (9) edge (10);
	\path (9) edge (12);
	\path (9) edge (13);
	\path (10) edge (11);
	\path (10) edge (13);
	\path (11) edge (14);
	\path (12) edge (13);
	\path (13) edge (14);
	\end{tikzpicture}  
\end{center}

\begin{proof}[Answer] Referenced \\
I will first begin by creating the lookup table for vertices $1-14$ as follows: \\

\begin{center}
\resizebox{20.21cm}{!}{
\begin{tabular}{ | c | c | c | c | c | c | c | c | c | c | c | c | c | c | c |}
 \hline
 Vertices: & 1 & 2 &  3 & 4 & 5 & 6 & 7 & 8 & 9 & 10 & 11 & 12 & 13 & 14\\ 
 \hline
 $X_{j, 14}$ & $X_{2, 14} + X_{7, 14}$  & $X_{3, 14} + X_{7, 14} + X_{8, 14}$ & $X_{4, 14} + X_{8, 14}$ & $X_{5, 14} + X_{9, 14} + X_{10, 14}$ & $X_{6, 14} + X_{11, 14}$ & $X_{11, 14}$ & $X_{8, 14}$ & $X_{9, 14} + X_{12, 14}$ & $X_{10, 14} + X_{12, 14} + X_{13, 14} $ & $X_{11, 14} + X_{13, 14}$ & $X_{14, 14}$ & $X_{13, 14}$ & $X_{14, 14}$ & 1   \\  
 \hline
  $X_{j, 14}$ & $X_{2, 14} + X_{7, 14}$  & $X_{3, 14} + X_{7, 14} + X_{8, 14}$ & $X_{4, 14} + X_{8, 14}$ & $X_{5, 14} + X_{9, 14} + X_{10, 14}$ & $X_{6, 14} + X_{11, 14}$ & $X_{11, 14}$ & $X_{8, 14}$ & $X_{9, 14} + X_{12, 14}$ & $X_{10, 14} + X_{12, 14} + X_{13, 14} $ & $X_{11, 14} + X_{13, 14}$ & 1 & $X_{13, 14}$ & 1 & 1   \\  
 \hline
  $X_{j, 14}$ & $X_{2, 14} + X_{7, 14}$  & $X_{3, 14} + X_{7, 14} + X_{8, 14}$ & $X_{4, 14} + X_{8, 14}$ & $X_{5, 14} + X_{9, 14} + X_{10, 14}$ & $X_{6, 14} + 1$& 1 & $X_{8, 14}$ & $X_{9, 14} + X_{12, 14}$ & $X_{10, 14} + X_{12, 14} + 1$ & 2 & 1 & 1 & 1 & 1   \\  
  \hline
   $X_{j, 14}$ & $X_{2, 14} + X_{7, 14}$  & $X_{3, 14} + X_{7, 14} + X_{8, 14}$ & $X_{4, 14} + X_{8, 14}$ & $X_{5, 14} + X_{9, 14} + 2$ & 2 & 1 & $X_{8, 14}$ & $X_{9, 14} + 1$ & 4 & 2 & 1 & 1 & 1 & 1   \\  
   \hline
   $X_{j, 14}$ & $X_{2, 14} + X_{7, 14}$  & $X_{3, 14} + X_{7, 14} + X_{8, 14}$ & $X_{4, 14} + X_{8, 14}$ & 8 & 2 & 1 & $X_{8, 14}$ & 5 & 4 & 2 & 1 & 1 & 1 & 1   \\ 
   \hline
   $X_{j, 14}$ & $X_{2, 14} + X_{7, 14}$  & $X_{3, 14} + X_{7, 14} + 5$ & 13 & 8 & 2 & 1 & 5 & 5 & 4 & 2 & 1 & 1 & 1 & 1   \\ 
   \hline
   $X_{j, 14}$ & $X_{2, 14} + 5$  & 23 & 13 & 8 & 2 & 1 & 5 & 5 & 4 & 2 & 1 & 1 & 1 & 1   \\ 
   \hline
    $X_{j, 14}$ & 28  & 23 & 13 & 8 & 2 & 1 & 5 & 5 & 4 & 2 & 1 & 1 & 1 & 1   \\ 
    \hline
\end{tabular}
}
\end{center}
From the above the last line of our lookup table for the vertices from $1-14$ will be: \\
\begin{center}
1 = 28 \\
2 = 23 \\
3 = 13 \\
4 = 8 \\
5 = 2 \\
6 = 1 \\
7 = 5 \\
8 = 5 \\
9 = 4 \\
10 = 2 \\
11 = 1 \\
12 = 1 \\
13 = 1 \\
14 = 1 \\
\end{center}
I will now show my work for vertices $9 - 14$, such that:\\
For vertex 14 I know that since by the definition of the problem itself that a single vertex is considered to have a path of length 0. \\
For vertex 13 the work we have is: \\
\begin{align*}
X_{13, 14} &= X_{14, 14} \\
X_{13, 14} &= 1
\end{align*} 
For vertex 12 the work we have is: \\
\begin{align*}
X_{12, 14} &= X_{13, 14} \\
X_{12, 14} &= 1
\end{align*} 
For vertex 11 the work we have is: \\
\begin{align*}
X_{11, 14} &= X_{14, 14} \\
X_{11, 14} &= 1
\end{align*} 
For vertex 10 the work we have is: \\
\begin{align*}
X_{10, 14} &= X_{11, 14} + X_{13, 14} \\
&= X_{11, 14} + 1 \\
&= 1 + 1 \\
X_{10, 14} &= 2
\end{align*} 
For vertex 9 the work we have is: \\
\begin{align*}
X_{9, 14} &= X_{10, 14} + X_{12, 14} + X_{13, 14} \\
 &= X_{10, 14} + X_{12, 14} + 1 \\
 &= X_{10, 14} + 1 + 1 \\
 &= 2 + 1 + 1 \\
X_{9, 14} &= 4
\end{align*}
From the above I have found the total number of paths from from each node of $j$ to 14, $j \geq 1$. 
\end{proof}

\newpage
\subsection{Problem 23(b)}
\addcontentsline{toc}{subsection}{Problem 23(b)}

Consider the following directed acyclic graph (the same as in the previous problem). Use dynamic programming to fill in a \textbf{one-dimensional} lookup table that computes the length of the longest path from each node $j$ to 14, for $j \geq 1$. You may use the recurrence
\[
L[j] = \begin{cases}
0 & j = 14 \\
1 + \max\{L[k] : (j,k) \in E(G)\} & j < 14
\end{cases}.
\]

Note that a single vertex is considered a path of length $0$. \textbf{Fill in the lookup table for all vertices 1-14; and in addition, clearly show work for vertices 9-14}.

\begin{center}
	\begin {tikzpicture}[-latex ,auto ,node distance =2 cm and 3cm ,on grid ,
	semithick ,
	state/.style ={ circle ,top color =white , bottom color = processblue!20 ,
	draw,processblue , text=blue , minimum width =1 cm}]

	\node[state] (1) {$1$};
	\node[state] (2) [right = of 1] {$2$};
	\node[state] (3) [right = of 2] {$3$};
	\node[state] (4) [right = of 3] {$4$};
	\node[state] (5) [right = of 4] {$5$};
	\node[state] (6) [right = of 5] {$6$};
	
	\node[state] (7) [below = of 2] {7};
	\node[state] (8) [right = of 7] {8};	
	\node[state] (9) [right = of 8] {9};	
	\node[state] (10) [right = of 9] {10};	
	\node[state] (11) [right = of 10] {11};	
	
	\node[state] (12) [below = of 9] {12};
	\node[state] (13) [right = of 12] {13};
	\node[state] (14) [right = of 13] {14};
	
	\path (1) edge (2);
	\path (1) edge (7);
	\path (2) edge (3);
	\path (2) edge (7);
	\path (2) edge (8);
	\path (3) edge (4);
	\path (3) edge (8);
	\path (4) edge (5);
	\path (4) edge (9);
	\path (4) edge (10);
	\path (5) edge (6);
	\path (5) edge (11);
	\path (6) edge (11);
	\path (7) edge (8);
	\path (8) edge (9);
	\path (8) edge (12);
	\path (9) edge (10);
	\path (9) edge (12);
	\path (9) edge (13);
	\path (10) edge (11);
	\path (10) edge (13);
	\path (11) edge (14);
	\path (12) edge (13);
	\path (13) edge (14);
	\end{tikzpicture}  
\end{center}

\begin{proof}[Answer]Referenced \\
I will begin by first creating the lookup table to finds the longest path of each node from $j$ to 14, as follows: \\
\begin{center}
\resizebox{20.3cm}{!}{
\begin{tabular}{ | c | c | c | c | c | c | c | c | c | c | c | c | c | c | c |}
 \hline
 Vertices: & 1 & 2 &  3 & 4 & 5 & 6 & 7 & 8 & 9 & 10 & 11 & 12 & 13 & 14\\ 
 \hline
 $L[J]$ & $1 + max(L[2], L[7])$  & $1 + max(L[3], L[7], L[8])$  & $1 + max(L[4], L[8])$  & $1 + max(L[5], L[9], L[10])$ & $1 + max(L[6], L[11])$  & $1 + max(L[11])$  & $1 + max(L[8])$  & $1 + max(L[9], L[12])$  & $1 + max(L[10], L[12], L[13])$  & $1 + max(L[11], L[13])$  & $1 + max(L[14])$  & $1 + max(L[13])$  & $1 + max(L[14])$ & 0   \\  
 \hline
 $L[J]$ & $1 + max(L[2], L[7])$  & $1 + max(L[3], L[7], L[8])$  & $1 + max(L[4], L[8])$  & $1 + max(L[5], L[9], L[10])$ & $1 + max(L[6], L[11])$  & $1 + max(L[11])$  & $1 + max(L[8])$  & $1 + max(L[9], L[12])$  & $1 + max(L[10], L[12], L[13])$  & $1 + max(L[11], L[13])$  & 1  & $1 + max(L[13])$  & 1 & 0   \\  
 \hline
  $L[J]$ & $1 + max(L[2], L[7])$  & $1 + max(L[3], L[7], L[8])$  & $1 + max(L[4], L[8])$  & $1 + max(L[5], L[9], L[10])$ & $1 + max(L[6], 1)$  & 2  & $1 + max(L[8])$  & $1 + max(L[9], L[12])$  & $1 + max(L[10], L[12], 1)$  & 2  & 1  & 2  & 1 & 0   \\  
  \hline
  $L[J]$ & $1 + max(L[2], L[7])$  & $1 + max(L[3], L[7], L[8])$  & $1 + max(L[4], L[8])$  & $1 + max(L[5], L[9], 2)$ & 3  & 2  & $1 + max(L[8])$  & $1 + max(L[9], 1)$  & 3  & 2  & 1  & 2  & 1 & 0   \\  
   \hline
  $L[J]$ & $1 + max(L[2], L[7])$  & $1 + max(L[3], L[7], L[8])$  & $1 + max(L[4], L[8])$  & 4 & 3  & 2  & $1 + max(L[8])$  & 4  & 3  & 2  & 1  & 2  & 1 & 0   \\  
   \hline
   $L[J]$ & $1 + max(L[2], L[7])$  & $1 + max(L[3], L[7], 4)$  & 5  & 4 & 3  & 2  & 5  & 4  & 3  & 2  & 1  & 2  & 1 & 0   \\  
   \hline
    $L[J]$ & $1 + max(L[2], 5)$ & 6 & 5  & 4 & 3  & 2  & 5  & 4  & 3  & 2  & 1  & 2  & 1 & 0   \\  
   \hline
    $L[J]$ & 7 & 6 & 5  & 4 & 3  & 2  & 5  & 4  & 3  & 2  & 1  & 2  & 1 & 0   \\  
    \hline
\end{tabular}
}
\end{center}
From the above I have created the one-dimensional lookup table that computes the longest path from each node $j$ to 14, I will now show my work for vertices $9 - 14$, such that: \\
For vertex 14, by the definition of our recurrence $L[j]$, we know that if j were to equal 14, than the path would be 0. \\
For vertex 13 the work we have is: \\
\begin{align*}
L[13] &= 1 + max(L[14]) \\
&= 1 + max(0) \\
&= 1 + 0 \\
L[13] &= 1
\end{align*}
For vertex 12 the work we have is: \\
\begin{align*}
L[12] &= 1 + max(L[13]) \\
&= 1 + max(1) \\
&= 1 + 1 \\
L[12] &= 2
\end{align*}
For vertex 11 the work we have is: \\
\begin{align*}
L[11] &= 1 + max(L[14]) \\
&= 1 + max(0) \\
&= 1 + 0 \\
L[11] &= 1
\end{align*}
For vertex 10 the work we have is: \\
\begin{align*}
L[10] &= 1 + max(L[11], L[13]) \\
&= 1 + max(1, 1) \\
&= 1 + 1 \\
L[10] &= 2
\end{align*}
For vertex 13 the work we have is: \\
\begin{align*}
L[9] &= 1 + max(L[10], L[12], L[13]) \\
&= 1 + max(2, 1, 1) \\
&= 1 + 2 \\
L[9] &= 3
\end{align*}
From the above I have found the longest path from from each node of $j$ to 14, $j \geq 1$. 
\end{proof}

%%%%%%%%%%%%%%%%%%%%%%%%%%%%%%%%%%%%%%%%%%%%%%%%%%

\end{document} % NOTHING AFTER THIS LINE IS PART OF THE DOCUMENT