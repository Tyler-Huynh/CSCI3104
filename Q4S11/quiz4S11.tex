\documentclass[11pt]{article} 
\usepackage[english]{babel}
\usepackage[utf8]{inputenc}
\usepackage[margin=0.5in]{geometry}
\usepackage{amsmath}
\usepackage{amsthm}
\usepackage{amsfonts}
\usepackage{amssymb}
\usepackage[usenames,dvipsnames]{xcolor}
\usepackage{graphicx}
\usepackage[siunitx]{circuitikz}
\usepackage{tikz}
\usepackage[colorinlistoftodos, color=orange!50]{todonotes}
\usepackage{hyperref}
\usepackage[numbers, square]{natbib}
\usepackage{fancybox}
\usepackage{epsfig}
\usepackage{soul}
\usepackage[framemethod=tikz]{mdframed}
\usepackage[shortlabels]{enumitem}
\usepackage[version=4]{mhchem}
\usepackage{multicol}

\usepackage{mathtools}
\usepackage{comment}
\usepackage{enumitem}
\usepackage[utf8]{inputenc}
\usepackage[linesnumbered,ruled,vlined]{algorithm2e}
\usepackage{listings}
\usepackage{color}
\usepackage[numbers]{natbib}
\usepackage{subfiles}
\usepackage{tkz-berge}


\newtheorem{prop}{Proposition}[section]
\newtheorem{thm}{Theorem}[section]
\newtheorem{lemma}{Lemma}[section]
\newtheorem{cor}{Corollary}[prop]

\theoremstyle{definition}
\newtheorem{definition}{Definition}

\theoremstyle{definition}
\newtheorem{required}{Problem}
\newtheorem*{requiredHC}{Problem HC}

\theoremstyle{definition}
\newtheorem{ex}{Example}


\setlength{\marginparwidth}{3.4cm}
%#########################################################

%To use symbols for footnotes
\renewcommand*{\thefootnote}{\fnsymbol{footnote}}
%To change footnotes back to numbers uncomment the following line
%\renewcommand*{\thefootnote}{\arabic{footnote}}

% Enable this command to adjust line spacing for inline math equations.
% \everymath{\displaystyle}

% _______ _____ _______ _      ______ 
%|__   __|_   _|__   __| |    |  ____|
%   | |    | |    | |  | |    | |__   
%   | |    | |    | |  | |    |  __|  
%   | |   _| |_   | |  | |____| |____ 
%   |_|  |_____|  |_|  |______|______|
%%%%%%%%%%%%%%%%%%%%%%%%%%%%%%%%%%%%%%%

\title{
\normalfont \normalsize 
\textsc{CSCI 3104 Fall 2022 \\ 
Instructors: Prof. Grochow and Chandra Kanth Nagesh} \\
[10pt] 
\rule{\linewidth}{0.5pt} \\[6pt] 
\huge Quiz 4 S11 \\
\rule{\linewidth}{2pt}  \\[10pt]
}
%\author{Your Name}
\date{}

\begin{document}
\definecolor {processblue}{cmyk}{0.96,0,0,0}
\definecolor{processred}{rgb}{200, 0, 0}
\definecolor{processgreen}{rgb}{0, 255, 0}
\DeclareGraphicsExtensions{.png}
\DeclareGraphicsExtensions{.gif}
\DeclareGraphicsExtensions{.jpg}

\maketitle


%%%%%%%%%%%%%%%%%%%%%%%%%
%%%%%%%%%%%%%%%%%%%%%%%%%%
%%%%%%%%%%FILL IN YOUR NAME%%%%%%%
%%%%%%%%%%AND STUDENT ID%%%%%%%%
%%%%%%%%%%%%%%%%%%%%%%%%%%
\noindent
Due Date \dotfill Thursday Oct 13, 2022 8pm MT \\
Name \dotfill \textbf{Tyler Huynh} \\
Student ID \dotfill \textbf{109603994} \\
Quiz Code (enter in Canvas to get access to the LaTeX template) \dotfill \textbf{ASIIO}


\tableofcontents

\section*{Instructions}
\addcontentsline{toc}{section}{Instructions}
 \begin{itemize}
	\item You may either type your work using this template, or you may handwrite your work and embed it as an image in this template. \textbf{If you choose to handwrite your work, the image must be legible, and oriented so that we do not have to rotate our screens to grade your work.} We have included some helpful LaTeX commands for including and rotating images commented out near the end of the LaTeX template.
	\item You should submit your work through the \textbf{class Gradescope page} only. Please submit one PDF file, compiled using this \LaTeX \ template.
	\item You may not need a full page for your solutions; pagebreaks are there to help Gradescope automatically find where each problem is. Even if you do not attempt every problem, please submit this document with no fewer pages than the blank template (or Gradescope has issues with it).

	\item You \textbf{may not collaborate with other students}. \textbf{Copying from any source is an Honor Code violation. Furthermore, all submissions must be in your own words and reflect your understanding of the material.} If there is any confusion about this policy, it is your responsibility to clarify before the due date. 

	\item Posting to \textbf{any} service including, but not limited to Chegg, Discord, Reddit, StackExchange, etc., for help on an assignment is a violation of the Honor Code.

	\item You \textbf{must} virtually sign the Honor Code (see Section \ref{HonorCode}). Failure to do so will result in your assignment not being graded.
\end{itemize}


\newpage
\section*{Honor Code (Make Sure to Virtually Sign)} \label{HonorCode}
\addcontentsline{toc}{section}{Honor Code (Make Sure to Virtually Sign)}
\hypertarget{HonorCode}{}

\begin{requiredHC}
\begin{itemize}
\item My submission is in my own words and reflects my understanding of the material.
\item Any collaborations and external sources have been clearly cited in this document.
\item I have not posted to external services including, but not limited to Chegg, Reddit, StackExchange, etc.
\item I have neither copied nor provided others solutions they can copy.
\end{itemize}

%\noindent In the specified region below, clearly indicate that you have upheld the Honor Code. Then type your name. 
\end{requiredHC}

\begin{proof}[I agree to the above, Tyler Huynh.]
%% Typing "I agree to the above," followed by your name is sufficient.
\end{proof}


\newpage
\setcounter{section}{10}
\section{Standard 11: Ford--Fulkerson max flow--min cut algorithm}

\setcounter{required}{10}
\begin{required} 
For Problem 11, consider the following flow network.

\begin{center}
	\begin {tikzpicture}[-latex ,auto ,node distance =2 cm and 3cm ,on grid ,
	semithick ,
	state/.style ={ circle ,top color =white , bottom color = processblue!20 ,
	draw,processblue , text=blue , minimum width =1 cm}]

	%%%%%%%%% upper path
	\node[state] (s) {$s$};
	\node[state] (A) [above right = of s] {$A$};
	\node[state] (B) [right = of A] {$B$};
	\node[state] (C) [right = of B] {$C$};
	\node[state] (D) [right = of C] {$D$};
	\node[state] (E) [right = of D] {$E$};

	\node[state] (t) [below right = of E] {$t$};

	\path (s) edge node[above] {$0 / 4$} (A);
	\path (A) edge node[above] {$0 / 3$} (B);
	\path (B) edge node[above] {$3 / 3$} (C);
	\path (C) edge node[above] {$3 / 3$} (D);
	\path (D) edge node[above] {$3 / 4$} (E);
	\path (E) edge node[above] {$3 / 4$} (t);


	%%%%%%% lower path
	\node[state] (F) [below right = of s] {$F$};
	\node[state] (G) [right = of F] {$G$};
	\node[state] (H) [right = of G] {$H$};
	\node[state] (I) [right = of H] {$I$};
	\node[state] (J) [right = of I] {$J$};
	
	
	\path (s) edge node[below left] {$3 / 4$} (F);
	\path (F) edge node[below] {$3 / 4$} (G);
	\path (G) edge node[below] {$3 / 3$} (H);
	\path (H) edge node[below] {$0 / 3$} (I);
	\path (I) edge node[below] {$0 / 3$} (J);
	\path (J) edge node[below right] {$0 / 4$} (t);
	
	%%%%%%%%% backwards edge
	\path (H) edge node[ right] {$ 3 / 3$} (B);
	
	\end{tikzpicture}  
\end{center}

\end{required}

\renewcommand{\theenumi}{\alph{enumi}}
\begin{enumerate}
\item Specify an augmenting path (from $s$ to $t$). No explanation needed, just specify the path.
\begin{proof}[Answer]
$s \to A \to B \to H \to I \to J \to t$
\end{proof}

\vfill 
\item What is the maximum amount of flow you can push along your augmenting path? Clearly specify what calculation you do to figure this out, and write out your work/calculation.
\begin{proof}[Answer]
The maximum amount of flow that I can push through the path of $s \to A \to B \to H \to I \to J \to t$, is 3. This is the maximum amount of flow because on the edge of $B \to H$ we are receiving 3 flow from H, thus we can also push 3 flow back to H. On the edges of $(s \to A), (A \to B), (H \to I), (I \to J), and (J \to t)$ the capacties of these edges are empty. Thus we can push a maximum flow of 3 along this flow augmenting path.
\end{proof}

\vfill 
\item Push the amount of flow from part (b) along the path you found in part (a), and draw the updated flow network here.
\begin{proof}[Answer]


\begin{center}
	\begin {tikzpicture}[-latex ,auto ,node distance =2 cm and 3cm ,on grid ,
	semithick ,
	state/.style ={ circle ,top color =white , bottom color = processblue!20 ,
	draw,processblue , text=blue , minimum width =1 cm}]

	%%%%%%%%% upper path
	\node[state] (s) {$s$};
	\node[state] (A) [above right = of s] {$A$};
	\node[state] (B) [right = of A] {$B$};
	\node[state] (C) [right = of B] {$C$};
	\node[state] (D) [right = of C] {$D$};
	\node[state] (E) [right = of D] {$E$};

	\node[state] (t) [below right = of E] {$t$};

	\path (s) edge node[above] {$3 / 4$} (A);
	\path (A) edge node[above] {$3 / 3$} (B);
	\path (B) edge node[above] {$3 / 3$} (C);
	\path (C) edge node[above] {$3 / 3$} (D);
	\path (D) edge node[above] {$3 / 4$} (E);
	\path (E) edge node[above] {$3 / 4$} (t);


	%%%%%%% lower path
	\node[state] (F) [below right = of s] {$F$};
	\node[state] (G) [right = of F] {$G$};
	\node[state] (H) [right = of G] {$H$};
	\node[state] (I) [right = of H] {$I$};
	\node[state] (J) [right = of I] {$J$};
	
	
	\path (s) edge node[below left] {$3 / 4$} (F);
	\path (F) edge node[below] {$3 / 4$} (G);
	\path (G) edge node[below] {$3 / 3$} (H);
	\path (H) edge node[below] {$3 / 3$} (I);
	\path (I) edge node[below] {$3 / 3$} (J);
	\path (J) edge node[below right] {$3 / 4$} (t);
	
	%%%%%%%%% backwards edge
	\path (H) edge node[ right] {$ 3 / 3$} (B);
	
	\end{tikzpicture}  
\end{center}


% YOUR ANSWER TO (c) HERE
%If you wish, you may un-comment the tikzpicture below (remove the \begin{comment} and \end{comment} lines) and modify
%the edge weights. You may also hand-draw and embed an image

%Include an Image: \includegraphics{ImageFileName}
%Include an Image and Rotate 90 degree: \includegraphics[angle=90]{ImageFileName}
%Include an Image, Rotate by 180 degrees, and scale by 50\% \includegraphics[scale=0.5, angle=90]{ImageFileName}

\begin{comment}
\begin{center}
	\begin {tikzpicture}[-latex ,auto ,node distance =2 cm and 3cm ,on grid ,
	semithick ,
	state/.style ={ circle ,top color =white , bottom color = processblue!20 ,
	draw,processblue , text=blue , minimum width =1 cm}]

	%%%%%%%%% upper path
	\node[state] (s) {$s$};
	\node[state] (A) [above right = of s] {$A$};
	\node[state] (B) [right = of A] {$B$};
	\node[state] (C) [right = of B] {$C$};
	\node[state] (D) [right = of C] {$D$};
	\node[state] (E) [right = of D] {$E$};

	\node[state] (t) [below right = of E] {$t$};

	\path (s) edge node[above] {$0 / 4$} (A);
	\path (A) edge node[above] {$0 / 3$} (B);
	\path (B) edge node[above] {$3 / 3$} (C);
	\path (C) edge node[above] {$3 / 3$} (D);
	\path (D) edge node[above] {$3 / 3$} (E);
	\path (E) edge node[above] {$3 / 4$} (t);


	%%%%%%% lower path
	\node[state] (F) [below right = of s] {$F$};
	\node[state] (G) [right = of F] {$G$};
	\node[state] (H) [right = of G] {$H$};
	\node[state] (I) [right = of H] {$I$};
	\node[state] (J) [right = of I] {$J$};
	
	
	\path (s) edge node[below left] {$3 / 4$} (F);
	\path (F) edge node[below] {$3 / 3$} (G);
	\path (G) edge node[below] {$3 / 3$} (H);
	\path (H) edge node[below] {$0 / 3$} (I);
	\path (I) edge node[below] {$0 / 3$} (J);
	\path (J) edge node[below right] {$0 / 4$} (t);
	
	%%%%%%%%% backwards edge
	\path (H) edge node[ right] {$ 3 / 3$} (B);
	
	\end{tikzpicture}  
\end{center}
\end{comment}

\end{proof}

\vfill 
\newpage
\item Calculate the value of the flow in your answer to part (c). Fill in the blanks (and add additional lines below as needed).

\begin{proof}
The value of the flow is the sum of the flows on the following edges: $(s \to A), (A \to B), (B \to H), (H \to I), (I \to J), and (J \to t)$ % YOU FILL IN HERE

The flow on edge $(s \to A)$ is 3 \\
The flow on edge $(A \to B)$ is 3 \\
The flow on edge $(B \to H)$ is 3 \\
The flow on edge $(H \to I)$ is 3 \\
The flow on edge $(I \to J)$ is 3 \\
The flow on edge $(J \to t)$ is 3 \\


Therefore the total flow value is the sum of the above, which is 18.
\end{proof}

\item (i) Specify the cut \emph{corresponding to your flow} (this is not just any cut with the same value as your flow---there is a specific procedure for finding the cut corresponding to your flow, see, e.g., Levet Section 5.4). (ii) Calculate the value of the cut. Specify where your numbers are coming from and what calculation you are doing. Show your work. 

\begin{proof}
The cut corresponding to my flow would be on the edges of $c\{A, B\}, c\{G, H\}, c\{B, H\}$. On these edges that capacities are full not allowing for any additional positive flow to be pushed through my flow. \\

The value of my cut would be 9. I found these numbers as by the definition of a minimum capacity cut it would be the sum of the capacities of the edges that are being cut. Such that: \\
\begin{center}
The edge $c\{A, B\} = 3$ \\
The edge $c\{G, H\} = 3$ \\
The edge $c\{B, H\} = 3$ \\
3 + 3 + 3 = 9 is the total value of our minimum capacity cut
\end{center}
From the above the total value of our minimum capacity cut is 9.
\vfill
\end{proof}

\end{enumerate}

% Either type your answer in above, or uncomment the \includegraphics command
% and use it to insert an approprate image. Try experimenting with the scale 
% 0.9 the width option to resize your image if necessary.

%\includegraphics[width=0.9\textwidth]{solution.jpg}


%%%%%%%%%%%%%%%%%%%%%%%%%%%%%%%%%%%%%%%%%%%%%%%%%%
\end{document} % NOTHING AFTER THIS LINE IS PART OF THE DOCUMENT



